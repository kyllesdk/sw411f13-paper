\section{Original Syntax Example 1} \label{first_syntax_example}
This is the first example of the original way to write code to Arduino. This example is a Hello World.
\begin{lstlisting}[caption=This is the first original Arduino example \cite{arduino_hello_world_example}, label=lst:first_syntax_example]
// include the library code:
#include <LiquidCrystal.h>

// initialize the library with the numbers of the interface pins
LiquidCrystal lcd(12, 11, 5, 4, 3, 2);

void setup() {
  pinMode(9, OUTPUT);
  analogWrite(9, 20);
  // set up the LCD's number of columns and rows: 
  lcd.begin(8, 2);
  // Print a message to the LCD.
  lcd.print("hello, world!");
  delay(2000);
}

void loop() {
  // set the cursor to column 0, line 0
  // (note: line 1 is the second row, since counting begins with 0):
  lcd.setCursor(0, 0);
  // print the number of seconds since reset:
  lcd.print(millis()/1000);
}
\end{lstlisting}
