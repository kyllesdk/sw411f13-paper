\section{Original Syntax Example 2}
This is the second example of the original way to write code to Arduino. This is about the joystick example.
\begin{lstlisting}[caption=This is the second original Arduino example \cite{arduino_joystick_example}, label=lst:second_syntax_example]
  // Declaration of Variables
  int ledPins [] = { 2,3,4,5,6,7,8,9 };    // Array of 8 leds mounted in a circle
  int ledVerde = 13;
  int espera = 40;                 // Time you should wait for turning on the leds
  int joyPin1 = 0;                 // slider variable connecetd to analog pin 0
  int joyPin2 = 1;                 // slider variable connecetd to analog pin 1
  int coordX = 0;                  // variable to read the value from the analog pin 0
  int coordY = 0;                  // variable to read the value from the analog pin 1
  int centerX = 500;               // we measured the value for the center of the joystick
  int centerY = 500;
  int actualZone = 0;
  int previousZone = 0;
  // Asignment of the pins
  void setup()
  {
    int i;
    beginSerial(9600);
    pinMode (ledVerde, OUTPUT);
    for (i=0; i< 8; i++)
    {
      pinMode(ledPins[i], OUTPUT);
    } 
  }
  // function that calculates the slope of the line that passes through the points
  // x1, y1 and x2, y2
  int calculateSlope(int x1, int y1, int x2, int y2)
  {
    return ((y1-y2) / (x1-x2));
  }
  // function that calculates in which of the 8 possible zones is the coordinate x y, given the center cx, cy
  int calculateZone (int x, int y, int cx, int cy)
  {
    int alpha = calculateSlope(x,y, cx,cy); // slope of the segment betweent the point and the center
    if (x > cx)
    {
      if (y > cy) // first cuadrant
      {
        if (alpha > 1) // The slope is > 1, thus higher part of the first quadrant
          return 0;
        else
          return 1;    // Otherwise the point is in the lower part of the first quadrant
      }
      else // second cuadrant
      {
        if (alpha > -1)
          return 2;
        else
          return 3;
      }
    }
    else
    {
      if (y < cy) // third cuadrant
      {
        if (alpha > 1)
          return 4;
        else
          return 5;
      }
      else // fourth cuadrant
      {
        if (alpha > -1)
          return 6;
        else
          return 7;
      }
    }
  } 
   void loop() {
    digitalWrite(ledVerde, HIGH); // flag to know we entered the loop, you can erase this if you want
    // reads the value of the variable resistors 
    coordX = analogRead(joyPin1);   
    coordY = analogRead(joyPin2);   
    // We calculate in which x
    actualZone = calculateZone(coordX, coordY, centerX, centerY); 
    digitalWrite (ledPins[actualZone], HIGH);     
    if (actualZone != previousZone)
      digitalWrite (ledPins[previousZone], LOW);
   // we print int the terminal, the cartesian value of the coordinate, and the zone where it belongs. 
  //This is not necesary for a standalone version
    serialWrite('C');
    serialWrite(32); // print space
    printInteger(coordX);
    serialWrite(32); // print space
    printInteger(coordY);
    serialWrite(10);
    serialWrite(13);
    serialWrite('Z');
    serialWrite(32); // print space
    printInteger(actualZone);
    serialWrite(10);
    serialWrite(13);
  // But this is necesary so, don't delete it!
    previousZone = actualZone;
   // delay (500);
 }
\end{lstlisting}
