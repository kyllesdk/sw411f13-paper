\section{Original Syntax Example 2}
This is the second example of the original way to write code to Arduino. This is about the LCD example.
\begin{lstlisting}[caption=This is the second original Arduino example \cite{arduino_LCD_example}, label=lst:second_syntax_example]
  #include <LiquidCrystal.h>

// initialize the library with the numbers of the interface pins
LiquidCrystal lcd(12, 11, 5, 4, 3, 2);
int count;

void setup() {
  pinMode(9, OUTPUT);
  analogWrite(9, 12);
  // set up the LCD's number of columns and rows: 
  lcd.begin(16, 2);
  // Print a message to the LCD.
  lcd.print("Foo Bar");
  count = 1;
  Serial.begin(9600);

  delay(2000);
}

void loop() {
  lcd.clear();
  delay(200);
  //If count divided by 3 and 5 equals 0 write Foo Bar
  if(count % 3 == 0 && count % 5 == 0){
    lcd.print("Foo Bar"); 
    Serial.println("Foo Bar"); 
  }
  else{
    //If count divided by 3 equals 0 write Foo 
    if(count % 3 == 0){
      lcd.print("Foo");
      Serial.println("Foo"); 
    }
    else{
      //If count divided by 5 equals 0 write Bar 
      if(count % 5 == 0){
        lcd.print("Bar");
        Serial.println("Bar"); 
      }
      else{
        //All other times write the number
        lcd.print(count);
        Serial.println(count); 
      }
    }
  }

  count++;
  if(count > 100){
    count = 2;
  }
  // delay at the end of the full loop:
  delay(1000);


}
\end{lstlisting}
