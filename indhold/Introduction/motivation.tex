\section{Motivation}\label{introduction:motivation}
The Arduino language is based on C and C++, which are languages that can be seen as cryptic and unintuitive to newly started programmers. An example of C not being intuitive is when creating an array to hold 3 elements, and the array has to be set to a size that can hold 4 elements. This is needed because there is a special sign to mark the end of an array (zerobit), which takes a slot itself. The programmer can work around it and/or create the functionality, but to the inexperienced programmer it adds complexity. Without prior experience with programming, C and C++ can be difficult languages to learn. \\
For beginners it could be better if the Arduino language, was simpler and easier to understand. The Arduino language is built upon C and C++, and therefore has similar flaws, at least from a beginners perspective. This project will look into giving beginners an alternative to the Arduino language by creating a new language along with a translator. A translator is needed in order to be recognizable as proper code for Arduino. This language will attempt to achieve user-friendliness with beginners in mind. A place to draw inspiration from is Python, which is a programming language praised by many experienced and inexperienced programmers for being intuitive and easy to learn. \cite{python:about} Python is a declarative language, which means that the code can be read logically. The programmer will discover that Python looks somewhat like English language. It is easier to understand because the code can be less cryptic compared to other languages. It is this kind of user friendliness that will be sought in this project.