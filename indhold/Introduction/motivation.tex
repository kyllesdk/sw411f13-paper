\section{Motivation}\label{introduction:motivation}
The Arduino language is based on C and C++, which are languages that can be seen cryptic and unintuitive to newly started programmers. C for example, has no boolean type, which means that the programmer has no option to define some element to be true of false. The programmer can work around it and/or create the functionality, but to the inexperienced programmer it adds to complexity. Another example of C not being intuitive is when creating an array to hold 3 elements, and the array has to be set to a size that can hold 4 elements. This is needed because there is a special sign to mark the end of an array, which takes a slot itself.  Without prior experience with programming, C and C++ can be difficult languages to learn. \\

It would be optimal if the Arduino language, was simple and easy to understand, since it is a good place to start for beignners. Unfortunately the Arduino language is built upon C and C++, and it therefore has similar flaws, at least from a beginners perspective. Therefore this project will look into giving beginners an alternative to Arduino language by creating a new language along with a translator for it, in order to be recognizable as proper code for Arduino. This language will attempt to achieve user-friendliness with beginners in mind. A place to draw inspiration from is Python, which is a programming language praised for being intuitive and easy to learn. \cite{python:about} Python is a declarative language which means that the code can be read logically. The programmer will discover that the code looks somewhat like the English language. It is easy to understand because the code can be less cryptic compared to other languages. It is this kind of user friendliness that will be sought in this project.