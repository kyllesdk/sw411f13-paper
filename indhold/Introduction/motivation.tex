\section{Motivation for this project}\label{introduction:motivation}
The Arduino language is based on C and C++, which are languages that can be seen as cryptic and unintuitive to programmers. C for example has no bool type, which means that the programmer has no option to define some element to be true of false. The programmer can work around it and/or create the functionality, but to the inexperienced programmer it adds to complexity. Another example of C not being intuitive is when creating an array to hold 3 elements, and the array has to be of size 4 because there is a special sign to mark the end of an array which takes a slot itself.  Without prior experiences with programming, C and C++ can be difficult languages to learn. \\

With the Arduino being a good place to start for a beginner, it would be optimal that the language designed for it was equally simple and easy to understand. Unfortunately as Wiring is built upon C and C++ it has similar flaws, at least from a beginners perspective. Therefore this project will look into giving beginners an alternative to Wiring by creating a new language along with a translator for it, in order to be recognizable as proper code for the Arduino. This language will attempt to achieve user friendliness with beginners in mind. A place to draw inspiration from is Python, which is a programming language praised for being intuitive and easy to learn. \todo{Kilde?} Python is a declarative language which means that code can be read logically. \todo{Kilde?} Thus the programmer might find that the code can look somewhat like the English language. It is easy to understand because the code can be less cryptic compared to other languages. It is this kind of user friendliness that will be sought in this project.