\section{Introduction}
When programming for the first time, code can seem unfriendly and hard to understand. A beginner can make mistakes and can have a hard time understanding how to code properly without a teacher. While it is the programmer who is not doing it right, perhaps the programmers way of doing it should be right. \\

Programming languages differ in various aspects and some seem more intuitive than others. Some languages are quite complex which often is related to a lot of features. For a beginner the complexity of having additional methods, to produce the same outcome or very specific code for uncommon tasks, is obsolete and can make a language hard to learn. \\

While a beginner can choose to start anywhere, one language that a beginner could encounter is the programming language of Arduino. With Arduino being a single small board with just one small processor for input and output, the options are relatively few and it is easy to get an overview. The programmer can use shields(attach other boards) to the Arduino, which will of course add to the complexity of what the programmer understanding what has to be done in order to achieve the desired outcome. Arduino gives beginners the option to play around with components.\\

The Arduino language is however based on C and C++, languages which can be seen as cryptic and unhelpful to the programmer. C for example has no boolean type, which means the programmer has no option to check for true/false. /todo{Ved vi det med sikkerhed? At der ikke findes bool ydtryk i C} It has to be created first. C can also seem counter-intuitive for a beginner. For example when creating an array to hold 3 characters, the array has to be of size 4.  Without experience, C and C++ can be hard languages to code. \\

With the Arduino being a good place to start for a beginner, it is not optimal that the programming language is not on par. \todo{Lidt tynd begrundelse om hvorfor Arduinos sprog ikke er begyndervenligt} Therefore this project will look into giving beginners an alternative to the Arduino language by creating a new language and a compiler for it. This language should attempt to achieve user friendliness for beginners. As such, a place to draw inspiration from is Python, which is a programming language praised for being intuitive and easy to learn. Python is declarative which means that code can be read logically. Thus the programmer might think that the code can look somewhat like normal English. It is easy to understand because what you write is what you get - it is not cryptic. It is this kind of user friendliness that will be sought in this project.