\section{Introduction}
When programming for the first time, most code can seem unintuitive and can be hard to understand. A beginner can make crucial mistakes and have a hard time learning how to program without having a teacher instructing them.\\

Programming languages differ in various aspects and some are more intuitive for beginners than others. Some languages are complex which often is due to the fact that they have a lot of features. For a beginner, the complexity of having several different methods to produce the same outcome, is obsolete and can make a language difficult to learn. \\

When programming for the first time, a programmer can choose to start with any language. A beginner with interest in learning about smaller hardware is likely to encounter the programming language of Arduino. Arduino gives programmers the option to program components and use them for a wide variety of purposes. Since Arduino is a small board with just one processor for input and output, the options are relatively few, and it is easy to get an overview. However the programmer can also attach other boards to the Arduino, which will add to the complexity of what the programmers has to understand in order to achieve the desired outcome. \\

The Arduino language is based on C and C++, which are languages that can be seen as cryptic and unintuitive to programmers. C for example has no boolean type, which means that the programmer has no option to check for true/false. /todo{Ved vi det med sikkerhed? At der ikke findes bool ydtryk i C} It has to be created first. C can also seem unintuitive for a beginner. For example when creating an array to hold 3 characters, the array has to be of size 4.  Without prior experiences with programming, C and C++ can be hard languages to learn. \\

With the Arduino being a good place to start for a beginner, it is not optimal that the programming language is not on par. \todo{Lidt tynd begrundelse om hvorfor Arduinos sprog ikke er begyndervenligt} Therefore this project will look into giving beginners an alternative to the Arduino language by creating a new language and a compiler for it. This language will attempt to achieve userfriendliness for beginners. A place to draw inspiration from is Python, which is a programming language praised for being intuitive and easy to learn. \todo{Kilde?} Python is declarative which means that code can be read logically. \todo{Kilde?} Thus the programmer might think that the code can look somewhat like the English language. It is easy to understand because what you write is what you get, it is not cryptic in any way. It is this kind of user friendliness that will be sought in this project.