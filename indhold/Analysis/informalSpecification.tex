\chapter{Informal Specification}\label{analysis:informal-specification}
The purpose of this section is to outline what the programming language BAL, proposed in this project, contains.
\\The language created in this project is a simplification of the Arduino language. Its purpose is to simplify the process of writing programs for Arduino.   

\subsection{Data Types}
The language contains the following data types: \\ 
\begin{center}
\begin{tabular}{ l l l l}
int & float & string & boolean \\
\end{tabular}
\end{center}

The different data types can hold different ranges or types of values. The values are specified more below: 
\begin{itemize}
\item \textbf{Int} or integer can hold numbers between $-2^{15}$ and $2^{15}-1$. Integers do not include floating point numbers.
\item \textbf{Float} allows decimals as opposed to integers and have a range from $1.175494351e-38$ to $3.402823466e+384$.
%\item \textbf{Double} is the numeric type in the source language with the biggest range, which goes from $2.2250738585072014e-308$ to $1.7976931348623157e+308$.
\item \textbf{String} can hold any kind of number and/or character. 
%\item \textbf{Array} can consist of collections of elements, either values or variables. For instance it can hold numbers and/or string of characters. 
\item \textbf{Boolean} can either be [1] (true) or [0] (false). 
\end{itemize}

\subsection{Keywords}
The following words are reserved words in the programming language:\\ 
\begin{center}
\begin{tabular}{ l l l l l l}
if & else & elseif & do & end & function \\
while & return & true & false & int \\
float & string & void & loop & setup \\
\end{tabular}
\end{center}

\subsection{Variables}
The variables in the language can have any name represented by upper- and/or lower case English letters. In addition, the underscore sign ``\_ '' can also be used to make more complex variable-names. 

\subsection{Encapsulation}
The encapsulation in the language is defined by \textbf{do} and \textit{end}. \textit{do} opens the encapsulation, and \textbf{end} closes it.   

\subsection{Loops and Conditions}
The language contains a \textit{while-loop} with a condition check before the loop is executed. The language also contains an \textit{if do else} condition. Additionally, \textit{else if} can be used.   

\subsection{White-space and Commentary}
The compiler ignores the white-spaces as well as anything to the right-hand side of a comment, marked by ``//''. Additionally a block comment, marked by ``/* (comment here) */''.


%\subsection{Type conversion}
%\todo{Matti: Vi skal have defineret type conversion}