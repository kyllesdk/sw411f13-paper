\chapter{Informal Specification}\label{analysis:informal-specification}
The purpose of this section is create an outline of what the programming language, created in this project, will contain.
\\The language created in this project is a simplification of the Arduino language. Its purpose is to simplify the process of writing programs for the Arduino.   

\subsection{Data types}
The language contains the following data types: 
\begin{itemize}
\item int
\item float
\item double
\item string
\item array
\item boolean
\end{itemize}

\subsection{Keywords}
The following words are reserved words in the programming language:\\
if, else, elsif, do, end, function, while, return, true, false, void, int, float, double, string, array, bool, print.

\subsection{Variables}
The variables in the language have to defined by using all upper- and/or lower case English letters. The underscore sign ``\_ '' can also be used. 

\subsection{Encapsulation}
The encapsulation in the language is defined by \textbf{do} and \textbf{end}. \textbf{do} opens the encapsulation, and \textbf{end} closes it.   

\subsection{Loops and conditions}
The language contains a \textbf{while-loop} with a condition check before the loop is executed. The language also contains an \textbf{if then else} condition. Additionally \textbf{else if} can be nested inside the \textbf{if then else}.   

\subsection{Type conversion}
\todo{Matti: Vi skal have defineret type conversion}