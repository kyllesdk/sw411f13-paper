\chapter{Informal Specification}\label{analysis:informal-specification}
The purpose of this section is to outline what the programming language, proposed in this project, contains.
\\The language created in this project is a simplification of the Arduino language. Its purpose is to simplify the process of writing programs for Arduino.   

\subsection{Data types}
The language contains the following data types: \\ 
\begin{center}
\begin{tabular}{ l l l l}
int & float & string & boolean \\
\end{tabular}
\end{center}

The different data types can hold different ranges or types of values. The values are specified more below: 
\begin{itemize}
\item \textbf{Int} or integer can hold numbers between $-32.768$ and $32.767$. Unsigned integers have a range from $0$ to $65.535$. Integers do not include floating point numbers. \todo{Morten: unsigned og signed?}
\item \textbf{Float} allows decimals as opposed to integers and have a range from $1.175494351e-38$ to $3.402823466e+384$.
%\item \textbf{Double} is the numeric type in the source language with the biggest range, which goes from $2.2250738585072014e-308$ to $1.7976931348623157e+308$.
\item \textbf{String} can hold any kind of number and/or character. \todo{Morten: Within which typeset and how long can the string be?}
%\item \textbf{Array} can consist of collections of elements, either values or variables. For instance it can hold numbers and/or string of characters. 
\item \textbf{Boolean} can either be ``true'' or ``false''. 
\end{itemize}

\subsection{Keywords}
The following words are reserved words in the programming language:\\ 
\begin{center}
\begin{tabular}{ l l l l l l}
if & else & elseif & do & end & function \\
while & return & true & false & void & int \\
float & system & string & out & bool & print\\
loop & setup \\
\end{tabular}
\end{center}

\subsection{Variables}
The variables in the language can have any name represented by upper- and/or lower case English letters. In addition, the underscore sign ``\_ '' can also be used to make more complex variable-names. 

\subsection{Encapsulation}
The encapsulation in the language is defined by \textbf{do} and \textbf{end}. \textbf{do} opens the encapsulation, and \textbf{end} closes it.   

\subsection{Loops and conditions}
The language contains a \textbf{while-loop} with a condition check before the loop is executed. The language also contains an \textbf{if then else} condition. Additionally \textbf{else if} can be nested inside the \textbf{if then else}.   

\subsection{White-space and commentary}
The compiler ignores the white-spaces as well as anything to the right-hand side of the comment, where ``//'' indicates a comment.


%\subsection{Type conversion}
%\todo{Matti: Vi skal have defineret type conversion}