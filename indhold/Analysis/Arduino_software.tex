\section{Arduino}
Arduino was created in Italy at the Interaction Design Institute Ivrea in Italy, by Massimo Banzi and David Cuartielles. They where looking for an easy and cheap way for design students to integrate micro controllers in the projects\cite{arduino:hist}. Both the board and the programming language was based on the works of Hernando Barragán, one of Massimo Banzi master thesis students \cite{Wiring:thesis}

\subsection{The hardware components}
Arduino is a single-board micro-controller\cite{Arduino}.
The boards were created to make a cheap way to control student-built interaction design projects. The board consist of open source hardware, which is designed around an 8-bit Atmel AVR micro-controller. Arduino boards varies in sizes. Arduino Uno board for example, has a max width of 2.1'' (5,33cm) and a length of 2.7'' (6,86cm). 

The board provides some input and output possibilities. However these varies depending on the board, though most have 14 digital I/O and 6 analog inputs. The I/O functions are placed on top of the board, are freely accessible, and consist of 0.1'' female headers. Besides the I/O there is also Power connector, which almost in all cases require 5 volt DC. There is an USB connection on the board, so that processing data to the micro-controller is possible, though it is shown as a virtual com-port on the connected computer. However on older boards, instead of the USB connection, a RS232 were used for serial communication. 

On the board there is an LED diode which is connected to the digital pin 13. When this diode is set to ``HIGH'' it will be turned on, and if its value is ``LOW'' it is off. Besides the LED diode there is also a reset button. If this button is pressed the micro-controller is reset. 

Arduino board gain extra features through shields. Shields is a ``board'', or rather an expansion board, which is plugged onto the Arduino board. This provides Arduino boards with the possibilities to control extra components for example motors, sensors and LCD displays.

\subsection{The language}
The language is based on wiring and therefore there are a lot of similarities between the to languages, but the Arduino team have added to, and improved the functions, and made it compatible with a wider range of chips. Also, both languages are implemented as a  version of C/C++, and are using an IDE based on the processing IDE \cite{Wiring:thesis}\cite{Arduino:IDE}.\\

\begin{tabular}{c | c}
\begin{lstlisting}
int ledPin = 8;  // LED connected to digital pin 8

void setup()
{
  pinMode(ledPin, OUTPUT);  // set ledPin pin as output
}

void loop()
{
  digitalWrite(ledPin, HIGH);  // set the LED on
  delay(1000);                 // wait for a second
  digitalWrite(ledPin, LOW);   // set the LED off
  delay(1000);                 // wait for a second
}
\end{lstlisting}
&
\begin{lstlisting}
int led = 13;

void setup() {                
  pinMode(led, OUTPUT);     
}

// the loop routine runs over and over again forever:
void loop() {
  digitalWrite(led, HIGH);   // turn the LED on (HIGH is the voltage level)
  delay(1000);               // wait for a second
  digitalWrite(led, LOW);    // turn the LED off by making the voltage LOW
  delay(1000);               // wait for a second
}
\end{lstlisting}
\end{tabular}
To help facilitate the compatibility with a wide range of AVR chips, the Arduino language makes great use of AVR Libc \cite{AVR:lib}, which is an open source C library to that supply the necessary functionality to make it possible to use the Atmel AVR micro controllers.