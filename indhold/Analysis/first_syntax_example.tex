\section{The first syntax example}
This is the first example based on the new language for Arduino. The code written in the Arduino language can be seen in Appendix \ref{first_syntax_example}, which, if they are compared, shows the difference between the new language and the original one. This first code example shows how the "hello, world!" is done in the source language. 
\begin{lstlisting}[caption=Hello World code example based on the source language, label=lst:syntax1]
// include the library code:
#include <LiquidCrystal.h>

// initialize the library with the numbers of the interface pins
LiquidCrystal lcd(12, 11, 5, 4, 3, 2);

void setup() begin
  // set up the LCD's number of columns and rows: 
  lcd.begin(16, 2);
  // Print a message to the LCD.
  lcd.print("hello, world!");
end

void loop() begin
  // set the cursor to column 0, line 1
  // (note: line 1 is the second row, since counting begins with 0):
  lcd.setCursor(0, 1);
  // print the number of seconds since reset:
  lcd.print(millis()/1000);
end
\end{lstlisting}
