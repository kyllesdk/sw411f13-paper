\section{The first syntax example}\label{sec:code_examples}
This section contains the first code example that is written in the new language for Arduino. The code written in the Arduino language can be seen in Appendix \ref{first_syntax_example}. If compared, it shows the difference between the new language, and the original one. This first code example shows how the "Hello, world!" program is written in the new language. "Hello, world!" is a simple test program that prints a string, which contains the text "Hello, world!".
 
\begin{lstlisting}[caption=Hello World code example based on the source language, label=lst:syntax1]
// include the library code:
#include <LiquidCrystal.h>

// initialize the library with the numbers of the interface pins
LiquidCrystal lcd(12, 11, 5, 4, 3, 2);

void setup() begin
  pinMode(9, OUTPUT);
  analogWrite(9, 20);
  // set up the LCD's number of columns and rows: 
  lcd.begin(8, 2);
  // Print a message to the LCD.
  lcd.print("hello, world!");
  delay(2000);
end

void loop() begin
  lcd.clear();
  // print the number of seconds since reset:
  lcd.print(millis()/1000);
  delay(1000);
end
\end{lstlisting}
