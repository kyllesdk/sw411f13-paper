\section{Priority Table}
In this section a priority table of the syntax is established for the source language. It contains level hierarchy (which operator is read first), which symbol is used for the operator, a small description of the operator, and the associativity of the operator. Associativity can be read differently, different ways of reading it produces different outcomes. For instance, $a * (b * c) = (a * b) * c$.
\begin{table}[H]
	\center
	\begin{tabular}{|l|l|l|l|}
	\hline
	\emph{Level} & \emph{Symbol} & \emph{Description} & \emph{Associativity} \\ 
 		\hline
 		1 & not & Logical negation & Left to right \\
 		& * & Multiplication & Left to right\\
 		& / & Division & Left to right\\
 		& \% & Modulus & Left to right\\
 		& sqrt() & Square root & \\
 		& $\texttt{\^{}}$ & Exponentiation &\\
 		\hline
 		2 & + & Addition & Left to right \\
 		& - & Subtraction & Left to right\\
 		\hline
 		3 & <  & Less than & Left to right \\
 		& > & Greater than & Left to right \\
 		& <= & Less than or equal to & Left to right \\
 		& >= & Greater than or equal to & Left to right \\
 		& equals & Equal to & Left to right \\
 		& notequals & Not equal to & Left to right \\
 		& or & Logical or & Left to right \\
 		& and & Logical and & Left to right\\
 		\hline
 		4 & = & Assignment & Right to left \\
 		\hline
	\end{tabular}
	\label{tab:priority}
	\caption{Table of operator priority}
\end{table} 
As seen in table \ref{tab:priority}, on the first level of the source language is negation, multiplication, division, modulus, square root, and exponentiation, this has been done because that is the priority of mathematics. Level two contains addition and subtraction which are placed just above the boolean expressions in order to have arithmetic and boolean expressions in the same statement but still separate them in order. The last level is assignment.