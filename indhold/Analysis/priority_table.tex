\section{Priority Table}
In this section a priority table of the syntax for the language is described. 
The priority table contains different sections, the first is level hierarchy. Level hierarchy describes the operators precedence. The operators with the highest precedence are evaluated first, and the operators with the lowest precedence are evaluated last. The level 1 operators have the highest precedence, and the level 4 operators have the lowest.
 
The table also shows which symbol is used for the different operators, and a short description of the operator. Furthermore it also describes the associativity of the operator. Associativity describes which direction an operator is evaluated in, either left to right or right to left.

\begin{table}[H]
	\center
	\begin{tabular}{|l|l|l|l|}
	\hline
	\emph{Level} & \emph{Symbol} & \emph{Description} & \emph{Associativity} \\ 
 		\hline
 		1 & not & Logical negation & Left to right \\
 		& * & Multiplication & Left to right\\
 		& / & Division & Left to right\\
 		& \% & Modulus & Left to right\\
 		& sqrt() & Square root & \\
 		& $\texttt{\^{}}$ & Exponentiation &\\
 		\hline
 		2 & + & Addition & Left to right \\
 		& - & Subtraction & Left to right\\
 		\hline
 		3 & <  & Less than & Left to right \\
 		& > & Greater than & Left to right \\
 		& <= & Less than or equal to & Left to right \\
 		& >= & Greater than or equal to & Left to right \\
 		& equals & Equal to & Left to right \\
 		& notequals & Not equal to & Left to right \\
 		& or & Logical or & Left to right \\
 		& and & Logical and & Left to right\\
 		\hline
 		4 & = & Assignment & Right to left \\
 		\hline
	\end{tabular}
	\label{tab:priority}
	\caption{Table of operator priority}
\end{table} 

As seen in table \ref{tab:priority}, on level 1 of the language is negation, multiplication, division, modulus, square root, and exponentiation. This precedence has been chosen because it matches that of mathematics. Level 2 contains addition and subtraction, which are placed above the boolean expressions in precedence. This is done in order to have arithmetic and boolean expressions in the same statement, but separate them in precedence. The last level is assignment.

\todo{Matti: This chapter is missing a section about how encapsulation is handled in our language, which is what scope rules is really all about. Consider moving priority table and environment-store-model to another chapter, unless there's a good reason for them to be in this chapter?}
