\chapter{Scope rules}\label{chap:scope-rules}
This section will be describing scope rules and the environment-store model for the source language. Both are essential for the source language. It is important to know how variables are being stored as well as knowing how the source language is operating with its blocks. 

\section{The environment-store model}\label{sec:es-model}
The environment-store model is important to know how works. The model shows how variables is stored, what variable is stored on what location, and with what value. By taking a look at figure \ref{fig:esmodel}, one will see the environment-store model. The model shows 3 boxes. The left box is the environment, the middle is the location and the right is the store. The environment is where variables are, the value of these variables has to be stored on a location, which is illustrated by the arrow, $env_v$ (variable-environment). On this location the value of the variable can be stored, which is illustrated by the $sto$ arrow. The model shows that the variable $x$ can be found in location 25, which has the value 5 stored. The same principle is valid for $y$ and $z$, however these 2 variables shares the same location, which has the same value stored. This means that the value in store is updated if for example, the value 13 gets changed to 15, then the value in both $y$ and $z$ will change. 
\begin{figure}[H]
\includegraphics{billeder/environment_store_model.png}
\caption{Environment-store model}
\label{fig:esmodel}
\end{figure}


\section{The scope rules}\label{sec:scope-rules}




\begin{center}
\begin{tabular}{ l l}
\hline
& \\
$[CALL-STAT-STAT_{BSS}]$ & $env'_{v}~[next \rightarrow l], env'_{p}~ \vdash \langle S,sto \rangle \rightarrow sto' \over env_{v}, ~ env_{p} \vdash \langle call~p, sto \rangle \rightarrow sto'$ \\
& where $env_{p}p = (S,env'{v},env'{p})$ \\
& and $l = env_{v}next$ \\
\hline
\end{tabular}
\end{center}







