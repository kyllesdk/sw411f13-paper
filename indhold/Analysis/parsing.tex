\section{Parsing}
This section contains a description of the parsing phase of the compiler. 
\\Parsing is the part of the syntax analysis that builds a parse tree and checks the input for syntax errors. The parser handles the token stream generated by the scanner by grouping the tokens into sentences based on the context free grammar. There are two different categories of parsers, top-down and bottom-up parsers.

\subsection{Top-Down Parsing}
This category of parsers is called top-down, because the parser begins with occurrences of the start symbol of the grammar and creates a parse tree of the input, by traveling from its root to its leaves. These parsers read the token stream from left to right, and derives productions using leftmost derivation. Two forms of top-down parsers are described below. 

\begin{itemize}
\item \textbf{Recursive-descent parsers}: This type of parser uses recursive procedures to parse a string.  In general each procedure uses one production rule of the grammar.
\item \textbf{Table-driven LL parsers}: This type of parser uses generic LL(k) parsing engines, and a parse table to keep control of which production rules to use. LL denotes the fact that it scans the input from left to right, and produces a leftmost derivation. Furthermore, it uses \textit{k} symbols of lookahead. Lookahead is explained in \ref{sec:lookahead}.
\end{itemize}

\subsection{Bottom-up Parsing}
This category of parsing is the opposite of top-down parsing. Bottom-up parsers are more powerful and efficient than top-down because they handle certain features better, such as recursive productions and common prefixes. It is called bottom-up, because the parser begins with occurrences of the start symbol of the grammer, and creates a parse tree of the input by traveling from its leaves to its root. These parsers read the token stream from left to right, and derives productions using rightmost derivation in reverse. Bottom-up parsers uses the grammar rules to replace the right-hand side of the production with its left-hand side. Top-down parsers does the opposite, by replacing the left-hand side of the production with its right-hand side. The type of bottom-up parser is described below. 

\begin{itemize}
\item \textbf{LR(k) parsers}: The LR parser read the token stream from left to right, and traces a rightmost derivation in reverse. As with LL parsers, the LR parser also uses \textit{k} symbols of lookahead.  
\end{itemize}
