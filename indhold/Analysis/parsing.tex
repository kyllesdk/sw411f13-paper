\section{Parsing}
This section will contain a describtion of the parsing phase of the compiler. 
\\Parsing is the part of the syntax analysis that builds a parse tree, and checks the input for a correct syntax. The parser handles the token stream generated by the scanner by grouping the tokens into sentences based on the context free grammar. There are two different main categories of parsers, top-down and bottom-up

\subsection{Top-Down Parsing}
This category of parsers of is called top-down, because the parser begins with the grammars start symbol and creates a parse tree, by going from its root to its leaves. These parsers read the token stream from left to right, and derives productions using leftmost derivation. Two forms of top-down parsers are described below. 

\begin{itemize}
\item \textbf{Recursive-descent parsers}: This form of parser uses recurisve procedures to parse a string. 
\item \textbf{Table-driven LL parsers}: This form of parser uses generic LL(k) parsing engines, and a parse table to keep control of which production rules to use. LL means that it scans the input from left to right, and produces a leftmost derivation. Furthermore it uses \textit{k} symbols of lookahead. 
\end{itemize}

\subsection{Bottom-up Parsing}
This category of parsing is the opposite of top-down parsing. Bottom-up parsers are more powerful and efficient than top-down, since it handles certain features better, such as recursive productions and common prefixes. It is called bottom-up, because the parser begins with the grammars start symbol and creates a parse tree, by going from its leaves to its root. These parsers read the token stream from left to right, and derives productions using rightmost derivation in reverse. Bottom-up parsers uses the grammar rules to replace the productions right-hand side with its left-handside. Top-down does parsers does the opposite. A form of bottom-up parser is described below. 

\begin{itemize}
\item \textbf{LR(k) parsers}: The LR parser read the token stream from left to right, and traces a rightmost derivation in reverse. There are different variants of LR parsers, such as LALR parsers and the SLR parsers. As with LL parsers, the LR parser also uses \textit{k} symbols of lookahead.  
\end{itemize}
