\section{The Arduino language}
The Arduino language is based on Wiring and therefore there are a lot of similarities between the two languages, but the Arduino team have added to it, improved the functions, and made it compatible with a wider range of chips.Both the languages are implemented as versions of C/C++, and are using an IDE based on the processing IDE \cite{Wiring:thesis}\cite{Arduino:IDE}.\\

In table \ref{tabel:comparison}, the syntax of the two languages are almost identical, and it is only in the functions parameters, that there are any noticeable difference.\\ 
To help facilitate the compatibility with a wide range of AVR chips, the Arduino language makes great use of AVR Libc \cite{AVR:lib}, which is an open source C library that supplies the necessary functionality to make it possible to use the Atmel AVR micro controllers.\\

\begin{table}[H]
\centering
\begin{tabular}{cc}
Wiring 
& 
Arduino \\ 
\hline 
\begin{lstlisting}
int ledPin = 8;

void setup(){
  pinMode(ledPin, OUTPUT);
}
void loop(){
  digitalWrite(ledPin, HIGH);
  delay(1000);
  digitalWrite(ledPin, LOW);
  delay(1000);
}
\end{lstlisting}  
& 
\begin{lstlisting}
int led = 13;

void setup() {                
  pinMode(led, OUTPUT);     
}
void loop() {
  digitalWrite(led, HIGH);
  delay(1000);
  digitalWrite(led, LOW);
  delay(1000);
}
\end{lstlisting} 
\end{tabular} 
\caption{Code examples in both Wiring and Arduino to make a LED flash}
\label{tabel:comparison}
\end{table}
\todo{Kunne det ikk være brugbart at sætte et eksempel af C op ved siden af dem også?}

