\section{The Arduino language}
The Arduino language is based on Wiring, which is an open source development platform, and therefore there are a lot of similarities between the two languages, but the Arduino team have added to it, improved the functions, and made it compatible with a wider range of chips. Both languages are implemented as versions of C/C++.

In table \ref{tabel:comparison}, there is exemplified the use of the two languages. One can notice that they are almost identical, and it is only in the functions parameters that there are some syntactic difference.\\ 
To help facilitate the compatibility with a wide range of AVR chips, the Arduino language makes great use of AVR Libc \cite{AVR:lib}, which is an open source C library that supplies the necessary functionality to make it possible to use the Atmel AVR micro controllers.\\

\begin{table}[H]\scriptsize
\centering
\begin{tabular}{cc}
\begin{minipage}{7cm}
\begin{lstlisting}[caption=Wiring]
int ledPin = 8;

void setup(){
  pinMode(ledPin, OUTPUT);
}
void loop(){
  digitalWrite(ledPin, HIGH);
  delay(1000);
  digitalWrite(ledPin, LOW);
  delay(1000);
}
\end{lstlisting} 
\end{minipage}
 
& 
\begin{minipage}{7cm}
\begin{lstlisting}[caption=Arduino]
int led = 13;

void setup() {                
  pinMode(led, OUTPUT);     
}
void loop() {
  digitalWrite(led, HIGH);
  delay(1000);
  digitalWrite(led, LOW);
  delay(1000);
}
\end{lstlisting}
\end{minipage}
 
\end{tabular} 
\caption{Code examples in both Wiring and Arduino to make a LED flash}
\label{tabel:comparison}
\end{table}

