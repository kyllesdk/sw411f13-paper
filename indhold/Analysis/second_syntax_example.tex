\section{The second syntax example}
This is the second example of how the new language looks like. Just like in the first syntax example \ref{lst:syntax1}, then this is an example derived from a code in the original Arduino language. The original code can be found in Appendix \ref{second_syntax_example}. This is second example is a small code doing a simple task. The task here is that when moving a joystick in one direction a LED will turn on, when moved in another direction another LED will be turned on, and so on. 
\begin{lstlisting}[caption=Joystick code example based on the source language, label=lst:syntax2]
    // Declaration of Variables
    // Array of 8 leds mounted in a circle
    int ledPins [] = { 2,3,4,5,6,7,8,9 };    
    
    int ledVerde = 13;
    int espera = 40		// Time waiting led to turn on
    int joyPin1 = 0		// slider variable to analog pin 0
    int joyPin2 = 1		// slider variable to analog pin 1
    int coordX = 0		// variable-read value from analog 0
    int coordY = 0		// variable-read value from analog 1
    int centerX = 500	// Measured center-value of joystick
    int centerY = 500;
    int actualZone = 0;
    int previousZone = 0;
    
    // Asignment of the pins
    void setup() begin
        int i;
        beginSerial(9600);
        pinMode (ledVerde, OUTPUT);
        while (i < 8) do
            pinMode(ledPins[i], OUTPUT);
            i = i + 1;
        end
    end

    // function that calculates the slope of the line that passes through the points
    // x1, y1 and x2, y2

    int calculateSlope(int x1, int y1, int x2, int y2) begin
    return ((y1-y2) / (x1-x2));
    end
    // function that calculates in which of the 8 possible zones is the coordinate x y, given the center cx, cy
    int calculateZone (int x, int y, int cx, int cy) begin
  
    int alpha = calculateSlope(x,y, cx,cy);     
    if (x > cx) do
        if (y > cy) 		// first quadrant 
            if (alpha > 1) do                   
                return 0;
            end
            else do
                return 1;                      
            end
        end
        else do				// second quadrant
            if (alpha > -1) do
                return 2;
            end
            else do
                return 3;
            end
        end
    end
    else do
        if (y < cy) do		// third quadrant
            if (alpha > 1) do
                return 4;
            end
            else do
                return 5;
            end
        end
        else do				// fourth quadrant
            if (alpha > -1) do
                return 6;
            end
            else do
                return 7;
            end
        end
    end
    
    void loop() begin
        digitalWrite(ledVerde, HIGH);           
                                                
        coordX = analogRead(joyPin1);   
        coordY = analogRead(joyPin2);   
                                                
        actualZone = calculateZone(coordX, coordY, centerX, centerY); 
        digitalWrite (ledPins[actualZone], HIGH);     
        if (actualZone != previousZone) do
            digitalWrite (ledPins[previousZone], LOW);
        end
                                                
        serialWrite('C');
        serialWrite(32); // print space
        printInteger(coordX);
        serialWrite(32); // print space
        printInteger(coordY);
        serialWrite(10);
        serialWrite(13);
        serialWrite('Z');
        serialWrite(32); // print space
        printInteger(actualZone);
        serialWrite(10);
        serialWrite(13);
                                                
        previousZone = actualZone;
        // delay (500);
    end
\end{lstlisting}
