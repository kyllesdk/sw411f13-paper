\section{The second syntax example}
This is the second example of how the new language looks like. Just like in the first syntax example \ref{lst:syntax1}, then this is an example derived from a code in the original Arduino language. The original code can be found in Appendix \ref{second_syntax_example}. This is second example is a small code doing a simple task.
It counts from 1 to 100 and prints a different output depending on the current count.\\
\begin{table}[H]
\centering
\begin{tabular}{|c|c|}
\hline 
Count & print \\ 
\hline 
Dividebel with 3 & Foo \\ 
\hline 
Dividebel with 5 & Bar \\ 
\hline 
Dividebel with 3 and 5 & FooBar \\ 
\hline 
Anything else & the count \\ 
\hline 
\end{tabular} 
\end{table}

So if the count is 9 it will print Foo and if it is 15 it will print FooBar. Despite it being a simple example it does make use of a lot of different operations.

\begin{lstlisting}[caption=LCD code example based on the source language, label=lst:syntax2]
// include the library code:
#include <LiquidCrystal.h>

// initialize the library with the numbers of the interface pins
LiquidCrystal lcd(12, 11, 5, 4, 3, 2);
int count;

void setup() begin
  pinMode(9, OUTPUT);
  analogWrite(9, 12);
  // set up the LCD's number of columns and rows: 
  lcd.begin(16, 2);
  // Print a message to the LCD.
  lcd.print("Foo Bar");
  count = 1;
  Serial.begin(9600);

  delay(2000);
end

void loop() begin
  lcd.clear();
  delay(200);
  //If count divided by 3 and 5 equals 0 write Foo Bar
  if(count % 3 equals 0 && count % 5 equals 0)then
      lcd.print("Foo Bar"); 
      Serial.println("Foo Bar"); 
  end
  //If count divided by 3 equals 0 write Foo 
  elseif(count % 3 equals 0)then
    lcd.print("Foo");
    Serial.println("Foo"); 
  end
    //If count divided by 5 equals 0 write Bar 
  elseif(count % 5 equals 0)then
    lcd.print("Bar");
    Serial.println("Bar"); 
  end
  else then
  //All other times write the number
    lcd.print(count);
    Serial.println(count); 
  end
end

  count = count + 1;
  if(count > 100)then
    count = 2;
  end
  // delay at the end of the full loop:
  delay(1000);


end
\end{lstlisting}
