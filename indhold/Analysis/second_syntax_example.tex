\section{The second syntax example}
This section contains the second example of a program written in the new language. This program is written based on a program from the original Arduino language. The original code can be found in Appendix \ref{second_syntax_example}. The following test program performs a simple task,
it counts from 1 to 100, by incrementing 1 at a time. Meanwhile it prints a different output depending on the current value of the counter.\\
\begin{table}[H]
\centering
\begin{tabular}{|c|c|}
\hline 
Count & print \\ 
\hline 
Dividebel with 3 & Foo \\ 
\hline 
Dividebel with 5 & Bar \\ 
\hline 
Dividebel with 3 and 5 & FooBar \\ 
\hline 
Anything else & the count \\ 
\hline 
\end{tabular} 
\end{table}

If the count is at 9, it will print Foo, and if the count is at 15, it will print FooBar. Despite it being a simple program, it makes use of a lot of different operators and expressions.

\begin{lstlisting}[caption=LCD code example based on the source language, label=lst:syntax2]
// include the library code:
#include <LiquidCrystal.h>

// instantiate the library with the numbers of the interface pins
instantiate LiquidCrystal lcd(12, 11, 5, 4, 3, 2);

void setup() do
  call pinMode(9, OUTPUT);
  call analogWrite(9, 12);
  // set up the LCD's number of columns and rows: 
  call lcd.begin(16, 2);
  // Print a message to the LCD.
  call lcd.print("Foo Bar");
  int count = 1;
  call delay(2000);
end

void loop() do
  int count = 0;
  call lcd.clear();
  call delay(200);
  //If count divided by 3 and 5 equals 0 write Foo Bar
  if(count % 3 EQUALS 0 AND count % 5 EQUALS 0) do
      call lcd.print("Foo Bar"); 
  end
  //If count divided by 3 equals 0 write Foo 
  elseif(count % 3 EQUALS 0) do
    call lcd.print("Foo");
  end
    //If count divided by 5 equals 0 write Bar 
  elseif(count % 5 EQUALS 0) do
    call lcd.print("Bar");
  end
  else do
  //All other times write the number
    call lcd.print(count);
  end
  
  int count = count + 1;
  if(count > 100)do
    var count = 2 + 3;
  end
  // delay at the end of the full loop:
  call delay(1000);
end
\end{lstlisting}
