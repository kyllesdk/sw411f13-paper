\section{Syntax definition}\label{sec:anlysis:syntax-definition}
\begin{lstlisting}[mathescape, captionpos=b, caption={Syntax formation rules}, label={lst:syntax-formation}]
$n$ $\in$ Num - Numerals
$x$ $\in$ Var - Variables
$a$ $\in$ Aexp - Arithmetic expressions
$b$ $\in$ Bexp - Boolean expressions
$S$ $\in$ Stm - Statements
$DV \in$ DecV - Variable declarations
$S_b$ $\in$ Substatement

$a$ ::= $n$ | $x$ | $a_1 + a_2$ | $a_1 - a_2$ | $a_1 * a_2$ | $a_1 / a_2$ | sqrt($a$) | $a_1$^$a_2$ | $a_1$ % $a_2$ | ($a$)

$b$ ::= "$a_1$ EQUALS $a_2$" | "$a_1 > a_2$" | "$a_1 < a_2$" | "$a_1 <= a_2$" | "$a_1 >= a_2$" | "$a_1$ !EQUALS $a_2$" | !$b$ | $b$ AND $b$ | $b$ OR $b$ | ($b$)

$S$ ::= "$x$ = $a$;" | if $b$ do $S$ end $S_b$ | while $b$ do $S$ end

$S_b$ ::= {elseif $b$ do $S_b$ end} | else do $S$ end | $\lambda$

$Dv$ ::= $x$ = $a$; Dv | $\lambda$
\end{lstlisting}

\subsection{Numerals}

\subsection{Variables}
%Variables has not been specified in the syntax formation rules, listing \ref{lst:syntax-formaton}, because variables is considered to be everything that is not specified in the rules above.

\subsubsection{Variable declarations}
%Though variables has not specified in the rules the rule for declaration of a variable has been specified.

\subsection{Arithmetic expressions}
The source language features arithmetic expressions. The transition rules for the expressions can be read in table \ref{tab:semantics_aritmethic}.
The rules listed in listing \ref{lst:syntax-formation} states that arithmetic expressions in the source language can consist of the following operations:
\begin{itemize}
	\item X - Variable
	\item N - Numeral
	\item Plus (+) - Addition between two numbers.
	\item Minus (-) - Subtraction between two numbers.
	\item Multiply (*) - Multiplication between two numbers.
	\item Division (/) - Division between two numbers.
	\item Square root ($\sqrt{x}$) - Finds the square root of a number.
	\item Power (${a_1}^{a_2}$) - Allows a number to be powered by another number.
	\item Modulus (\%) - Allows modulus to be used, which gives the opportunity to find the reminder between two numbers.
	\item Parenthesis (($a$)) - Specifies that an expression can be surrounded by parenthesis.
\end{itemize}

\subsection{Boolean expressions}
The source language supports boolean expressions, which is expressions that evaluates to \textit{true} or \textit{false}. The boolean transition rules are listed in table \ref{tab:semantics_boolean}.
\begin{itemize}
	\item Equals to ($a_1$ EQUALS $a_2$) - $EQUALS$ allows the programmer to compare two values, which will evaluates to \textit{true} if they match; if not it evaluates to \textit{false}.
	\item Greater than ($a_1 > a_2$) - ``$>$'' checks if one value is greater than another value. If $a_1$ is greater than $a_2$ it evaluates to \textit{true}, else it evaluates to \textit{false}.
	\item Less than ($a_1 < a_2$) - ``$<$'' checks if one value is lesser than another value. If $a_1$ is less than $a_2$ it evaluates to \textit{true}, else \textit{false}.
	\item Greater than or equal to ($a_1 >= a_2$) - 
	\item Less than or equal to ($a_1 <= a_2$) - 
	\item Different from ($a_1$ !EQUALS $a_2$) - 
	\item Not (!$b$) -
	\item And ($b$ AND $b$) -
	\item Or ($b$ OR $b$) -
	\item Parenthesis (($b$)) - Allows the boolean expressions to be insulated in parenthesis. The parenthesis will then evaluate to \textit{true} or \textit{false}.
\end{itemize}
\subsection{Statements}

\subsubsection{Substatements}


