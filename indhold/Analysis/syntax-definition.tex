\section{Syntax definition}\label{sec:anlysis:syntax-definition}
\todo{Intro goes here}
\begin{lstlisting}[mathescape, captionpos=b, caption={Syntax formation rules}, label={lst:syntax-formation}]
$n$ $\in$ Num - Numerals
$x$ $\in$ Var - Variables
$a$ $\in$ Aexp - Arithmetic expressions
$b$ $\in$ Bexp - Boolean expressions
$S$ $\in$ Stm - Statements
$Dv \in$ DecV - Variable declarations
$S_b$ $\in$ Substatement

$a$ ::= $n$ | $x$ | $a_1 + a_2$ | $a_1 - a_2$ | $a_1 * a_2$ | $a_1 / a_2$ | sqrt($a$) | $a_1$^$a_2$ | $a_1$ % $a_2$ | ($a$)

$b$ ::= "$a_1$ equals $a_2$" | "$a_1 > a_2$" | "$a_1 < a_2$" | "$a_1 <= a_2$" | "$a_1 >= a_2$" | "$a_1$ notequals $a_2$" | not$b$ | $b_1$ and $b_2$ | $b_1$ or $b_2$ | ($b$)

$S$ ::= $Dv$ | if $b$ do $S$ end $S_b$ | while $b$ do $S$ end | $S_1 ; S_2$

$S_b$ ::= {elseif $b$ do $S$ end} | else do $S$ end

$Dv$ ::= $x$ = $a$; Dv
\end{lstlisting}

\subsection{Numerals}
\todo{Description of numerals is needed.}
\subsection{Variables}
Variables has not been specified in the syntax definition, because it supports a very broad variety of things. A variable in the language can consist of text, expressions, boolean expression, numerals etc. Therefore the variable have not been specified simply because it can be so many things.

\subsubsection{Variable declarations}\label{sec:analysis:syntax-definition:variable-declaration}
Though variables has not specified in the syntax definition, the rule for declaration of a variable has been specified. The rule states that a variable has to be a variable on the left side and then the assignment character(=) which makes the right side assigned to the left side. The ``;'' character closes the variable declaration.

\subsection{Arithmetic expressions}
The source language features arithmetic expressions. The transition rules for the expressions can be read in table \ref{tab:semantics_aritmethic}.
The rules listed in listing \ref{lst:syntax-formation} states that arithmetic expressions in the source language can consist of the following operations:
\begin{itemize}
	\item X - Variable
	\item N - Numeral
	\item Plus (+) - Addition between two numbers.
	\item Minus (-) - Subtraction between two numbers.
	\item Multiply (*) - Multiplication between two numbers.
	\item Division (/) - Division between two numbers.
	\item Square root ($\sqrt{x}$) - Finds the square root of a number.
	\item Power (${a_1}^{a_2}$) - Allows a number to be powered by another number.
	\item Modulus (\%) - Allows modulus to be used, which gives the opportunity to find the reminder between two numbers.
	\item Parenthesis (($a$)) - Specifies that an expression can be surrounded by parenthesis.
\end{itemize}

\subsection{Boolean expressions}
The source language supports boolean expressions, which is expressions that evaluates to \textit{true} or \textit{false}. The boolean transition rules are listed in table \ref{tab:semantics_boolean}.
\begin{itemize}
	\item Equals to ($a_1$ equals $a_2$) - $equals$ allows the programmer to compare two values, which will evaluates to \textit{true} if they match; if not it evaluates to \textit{false}.
	\item Greater than ($a_1 > a_2$) - ``$>$'' checks if one value is greater than another value. If $a_1$ is greater than $a_2$ it evaluates to \textit{true}, else it evaluates to \textit{false}.
	\item Less than ($a_1 < a_2$) - ``$<$'' checks if one value is lesser than another value. If $a_1$ is less than $a_2$ it evaluates to \textit{true}, else \textit{false}.
	\item Greater than or equal to ($a_1 >= a_2$) - ``$>=$'' checks if one value is greater then or equals another value. If $a_1$ is greater then or equals $a_2$ it evaluates to \textit{true}, else \textit{false}.
	\item Less than or equal to ($a_1 <= a_2$) - ``$<=$'' checks if one value is less then or equals another value. If $a_1$ is less then or equals $a_2$ it evaluates to \textit{true}, else \textit{false}.
	\item Different from ($a_1$ notequals $a_2$) - ``$notequals$'' checks if two values are different. If $a_1$ is different from $a_2$ it evaluates \textit{true}, else \textit{false}. 
	\item Not (not$b$) - ``$not$'' checks if a value is boolean value is false. If $b$ is false it evaluates \textit{true}, else false.
	\item And ($b_1$ and $b_2$) - ``$and$'' check if two boolean values are both \textit{true}. If both $b_1$ and $b_2$ are \textit{true} it evaluates \textit{true}, else \textit{false}. 
	\item Or ($b_1$ or $b_2$) - ``$or$'' checks if one of the boolean values are \textit{true}. If either $b_1$ or $b_2$ are \textit{true} it evaluates \textit{true}, else \textit{false}.
	\item Parenthesis (($b$)) - Allows the boolean expressions to be insulated in parenthesis. The parenthesis will then evaluate to \textit{true} or \textit{false}.
\end{itemize}
\subsection{Statements}

$S$ ::= "$x$ = $a$;" | if $b$ do $S$ end $S_b$ | while $b$ do $S$ end

\begin{itemize}
	\item Variable declaration - A statement can consist of a variable declaration, section \ref{sec:analysis:syntax-definition:variable-declaration}.
	\item If-statement - A statement can consist of a if-statement which is a condition. The statement is given a boolean expression and then it evaluates the boolean expression. If the boolean expression evaluates to true the program goes into the if-statement where it can call another statement or declare a variable etc. If the condition evalutates to false then the program checks if the if-statement has and ``elseif'' or and ``else'' statement and then evaluates that.
	\item While loop - The while loop takes a boolean expression as parameter. As long the boolean expression evaluates to true the code inside the loop is looping. When the boolean expression evaluates to false the loop stop.
\end{itemize}

\subsubsection{Substatements}
\todo{Add something here}

