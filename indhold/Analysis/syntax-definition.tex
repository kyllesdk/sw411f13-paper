\chapter{Syntax and Semantics}\label{analysis:syntax-and-semantics}
In this section a description of the syntax and semantics in the programming language is described.

\section{Syntax}
When describing a programming language the syntax of the language is given by the set of rules which defines the words (finite sequences of characters) that can be used for writing a program in that language. The syntax of the language is described using Backus-Naur Form (BNF) which provides the context free grammar of the language.
In this report an extended version of BNF, Extended Backus-Naur Form (EBNF), is used. The advantage of using EBNF is to describe the set of rules in a more compact form, and the ability to describe regular expressions in the context free grammar. EBNF does not enhance the descriptive power of BNF, it only increases the readability and the write-ability.

\section{Syntax definition}\label{sec:anlysis:syntax-definition}
Below is the syntax definition written in EBNF which shows the seven syntactic categories followed by the defined rules. As an example of how to read these rules, let $a$ be an arithmetic expression. By looking at the rule for $a$, this expression could for example be $a_1 + a_2$ meaning it is expanded to be one arithmetic expression added with another. These two expressions could then again be anything within the rule for $a$, for example the numerals 5 and 7. There are no rules for expanding a numeral, and as such the expression $a$ can be evaluated no further.
\begin{lstlisting}[mathescape, captionpos=b, caption={Syntax formation rules}, label={lst:syntax-formation}]
$n$ $\in$ Num - Numerals
$x$ $\in$ Var - Variables
$a$ $\in$ Aexp - Arithmetic expressions
$b$ $\in$ Bexp - Boolean expressions
$S$ $\in$ Stm - Statements
$Dv \in$ DecV - Variable declarations
$S_b$ $\in$ Substatement

$a$ ::= $n$ | $x$ | $a_1 + a_2$ | $a_1 - a_2$ | $a_1 * a_2$ | $a_1 / a_2$ | sqrt($a$) | $a_1$^$a_2$ | $a_1$ % $a_2$ | ($a$)

$b$ ::= "$a_1$ equals $a_2$" | "$a_1 > a_2$" | "$a_1 < a_2$" | "$a_1 <= a_2$" | "$a_1 >= a_2$" | "$a_1$ notequals $a_2$" | not$b$ | $b_1$ and $b_2$ | $b_1$ or $b_2$ | ($b$)

$S$ ::= $Dv$ | if $b$ do $S$ end $S_b$ | while $b$ do $S$ end | $S_1 ; S_2$

$S_b$ ::= {elseif $b$ do $S$ end} | else do $S$ end

$Dv$ ::= $x$ = $a$; Dv
\end{lstlisting}

\subsection{Numerals}
In our programming language there are three numeral systems called integers, doubles, and floats. These types are needed to represent integers (for example -17, -239586, 0, 237, 9001). and decimal numbers (such as -12.3456, -30.009, 90.01 and 990.1259). Float and double are both used for decimal numbers with double being more precise than float at the cost of taking up more space. The numeral systems are however finite as the numerals can not exceed the size allocated to each float, integer and double. As such the limitations are:
\begin{itemize}
\item Integers: From $-2^{n-1}$ to $2^{n-1}-1$
\item Floats: From $3.4E-38$ to $3.4E+38$
\item Doubles: From $1.7E-308$ to $1.7E+308$
\end{itemize}

\subsection{Variables}
A variable in the language ranges over text, expressions, boolean expression, numerals etc. Because the variables can so many things, they have not been specified any further. 

\subsection{Metavariables}
\todo{Distinction between variables and metavariables is needed, what is meta variables?}

\subsubsection{Variable declarations}\label{sec:analysis:syntax-definition:variable-declaration}
The following describes the specified rule for declaration a variable. The rule states that a variable has to be a variable name on the left side of the assignment character "=", and then the value on the right side is thereby assigned to the left side. The ``;'' character closes the variable declaration.

\subsection{Arithmetic expressions}
The source language features arithmetic expressions. The transition rules for the expressions can be read in table \ref{tab:semantics_aritmethic}.
The rules listed in listing \ref{lst:syntax-formation} states that arithmetic expressions in the source language can consist of the following operations:
\begin{itemize}
	\item x - Variable
	\item n - Numeral
	\item Addition "$a_1 + a_2$" - Addition between two numbers.
	\item Subtraction "$a_1 - a_2$" - Subtraction between two numbers.
	\item Multiplication "$a_1 * a_2$" - Multiplication between two numbers.
	\item Division "$a_1 / a_2$" - Division between two numbers.
	\item Square root "sqrt($a$)" - Finds the square root of a number.
	\item Power "$a_1$\texttt{\^{}}$a_2$" - Allows a number to be powered by another number.
	\item Modulus "$a_1$ \% $a_2$" - Allows modulus to be used, which gives the opportunity to find the remainder between two numbers.
	\item Parenthesis "($a$)" - Specifies that an expression can be surrounded by parenthesis.
\end{itemize}

\subsection{Boolean expressions}
The source language supports boolean expressions, which are expressions that evaluate to \textit{true} or \textit{false}. The boolean transition rules are listed in table \ref{tab:semantics_boolean}.
\begin{itemize}
	\item Equals to ($a_1$ equals $a_2$) - $equals$ allows the programmer to compare two values, which will evaluates to \textit{true} if they match; if not it evaluates to \textit{false}.
	\item Greater than ($a_1 > a_2$) - ``$>$'' checks if one value is greater than another value. If $a_1$ is greater than $a_2$ it evaluates to \textit{true}, else it evaluates to \textit{false}.
	\item Less than ($a_1 < a_2$) - ``$<$'' checks if one value is lesser than another value. If $a_1$ is less than $a_2$ it evaluates to \textit{true}, else \textit{false}.
	\item Greater than or equal to ($a_1 >= a_2$) - ``$>=$'' checks if one value is greater then or equals another value. If $a_1$ is greater then or equals $a_2$ it evaluates to \textit{true}, else \textit{false}.
	\item Less than or equal to ($a_1 <= a_2$) - ``$<=$'' checks if one value is less then or equals another value. If $a_1$ is less then or equals $a_2$ it evaluates to \textit{true}, else \textit{false}.
	\item Different from ($a_1$ notequals $a_2$) - ``$notequals$'' checks if two values are different. If $a_1$ is different from $a_2$ it evaluates \textit{true}, else \textit{false}. 
	\item Negation (not$b$) - ``$not$'' checks if a value is boolean value is false. If $b$ is false it evaluates \textit{true}, else false.
	\item Conjunction ($b_1$ and $b_2$) - ``$and$'' check if two boolean values are both \textit{true}. If both $b_1$ and $b_2$ are \textit{true} it evaluates \textit{true}, else \textit{false}. 
	\item Disjunction ($b_1$ or $b_2$) - ``$or$'' checks if one of the boolean values are \textit{true}. If either $b_1$ or $b_2$ are \textit{true} it evaluates \textit{true}, else \textit{false}.
	\item Parenthesis (($b$)) - Allows the boolean expressions to be insulated in parenthesis. The parenthesis will then evaluate to \textit{true} or \textit{false}.
\end{itemize}
\subsection{Statements}

$S$ ::= Dv | if $b$ do $S$ end $S_b$ | while $b$ do $S$ end | $S_1 ; S_2$

\begin{itemize}
	\item Variable declaration - A statement can consist of a variable declaration, section \ref{sec:analysis:syntax-definition:variable-declaration}.
	\item If-statement - A statement can consist of an if-statement which is a condition. If the boolean expression evaluates to true another statement is called or declare a variable etc. If the condition evalutates to false then the program checks if the if-statement has an ``elseif'' or an ``else'' statement and then evaluates that.
	\item While loop - The while loop will run if the boolean expression evaluates to true. As long as the boolean expression is true, the statements inside the loop will keep repeatedly execute over and over. When the boolean expression evaluates to false, the loop is exited.
	\item Composition - To make programs run statements sequentially in the source language.
\end{itemize}

\subsubsection{Substatements}

$S_b$ ::= {elseif $b$ do $S$ end} | else do $S$ end

\begin{itemize}
	\item Elseif - If an if-statement is reached the source language allows to follow up using an elseif statement. Elseif follow the same rules as if-statements, but can be run zero to many times.
	\item Else - If the boolean expression in an if-statement evaluates to false, an else statement can be reached which will execute the other statement.
\end{itemize}

\todo{Add something here}

