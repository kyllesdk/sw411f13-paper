\chapter{Design Criteria}\label{chap:design-criteria}
This section presents the design criteria for the source language, showing which features are reached. 

\section{The chosen criteria}
When designing a language it is possible to define design criteria of the language, which can provide knowledge of the usability of the language. The design criteria that were considered to evaluate the source language are listed in table \ref{chap:design-criteria}. The language does not fulfil all the criteria, since not all of them are relevant for this projects purpose. For instance the ``Cost'' criteria as well as the ``Others'' criteria, including their sub-criteria are of no use. However ``Readability'', ``Write-ability'' and ``Reliability'' are relevant for this project, and the focus will be on these 3, and their respective sub-criteria.

\begin{table}[H]
	\center
	\begin{tabular}{|l|l|}
	\hline
	\emph{Main criteria} & \emph{Sub-criteria} \\ 
 		\hline
 		Readability & Overall simplicity \\
 		 & Orthogonality \\
 		 & Data types and control statements \\
 		 & Syntax considerations \\
 		\hline
 		Write-ability & Simplicity and orthogonality  \\
 		 & Support for abstraction \\
 		 & Expressivity \\
 		\hline
 		Reliability & Type checking \\
 		 & Exception handling \\
 		 & Aliasing \\
 		 & Readability and write-ability \\
 		\hline
 		Cost & Training programmers to use the language \\
 		 & Writing programs \\
 		 & Compiling programs \\
 		 & Executing programs \\
 		 & Language implementation system \\
 		 & Reliability \\
 		 & Maintaining programs \\
 		\hline
 		Other & Portability \\
 		 & Generality \\
 		 & Well-definedness \\
 		\hline
	\end{tabular}
	\label{tab:design}
	\caption{Table of design criteria}
\end{table} 

\textbf{Readability:} \\
The readability criteria is one of the most important criteria for this language, because the language is developed with the focus on it being easy to read, and thereby easy to understand for beginners in programming. \\
\begin{itemize}
\item Overall simplicity:\\
The language has a manageable set of features, which provides the users with the basic neccessities to be able to write programs in a simple matter. The language being limited adds simplicity, since there is no operator overloading possible.
\item Orthogonality:\\
A language that has good orthogonality is one that has few or a limited amount of constructs where every possible combination is allowed. Too much orthogonality is not always good, as it can cause problems. For example, if the users have too many choices, it can be difficult to determine which solution is the best.\\
The language is not orthogonal, because not every possible combination is legal. However, it does not reject all combinations either.  
\item Data types and control statements:\\
The language has adequate data types as well as control statements. It does not support all the types featured in Arduino, but there are enough to fulfil the simple tasks. For example, the data type decimal is not supported, which is the preferred data type when writing programs that handle finances. In BAL decimal and doubles are not included, since float provides the language with sufficient expressivity.
\item Syntax Consideration:\\
The consideration of the syntax for the language is mostly to be found within the keywords. For instance ``AND'' is self-describing, whereas many programming languages use ``\&\&'' for the same effect. The keywords allowed in the language are easy to use, because they are close to English. For example, if ``expression(1) is true, OR if expression(2) is true''. The important word in this example is ``OR'', which is the keyword for how to distinguish between two boolean expressions.
\end{itemize}
\textbf{Write-ability:} \\
The write-ability criteria is important, because it shows the criteria of how the code should be written. 
\begin{itemize}
\item Simplicity and Orthogonality: \\
It is important for the language to have a small number of rules required to write code. The code has to be simple to write, and easy to use with low orthogonality.
\item Support for abstraction: \\
Some programming languages support abstraction. For example, supporting imaginary numbers. The language does not support abstraction. The language does not require the option of imaginary numbers, because it has no use in this project, and it achieves its goals without them.
\item Expressivity: \\
The language is not expressive. It does not support expressions like ``$i++$'' or predefined functions like ``$and~then$''. These expressions are used to simplify coding. For example, instead of writing ``$i++$'', one would have written ``$i = i + 1$''. However, by using ``$i++$'' the programmer would both be saving space and time. By including expressivity, it is likely that readability will decrease. 
\end{itemize}
\textbf{Reliability:} \\
It is important that the source language is reliable. If the source language keeps producing errors or mistakes, the user is less likely to use the programming language.
\begin{itemize}
\item Type checking: \\
It is important that the language gets checked for type errors, this happens during the code generation.
\item Exception handling: \\
There is no exception handling in the language. If there is an error, the program will not run. Exception handling would increase the complexity of the language significantly, therefore it is not included.
\item Aliasing: \\
It is possible to refer to function-names, as well as variable-names in the language. However, it is not possible to refer to the same slot in memory in two or more distinct ways, which makes the source language weak regarding aliasing. 
\item Readability and write-ability:
In regards of readability and write-ability, it is important that the source language supports a natural way of writing and reading code. 
\end{itemize}
It is difficult to be completely objective regarding the design criteria, because they can be seen from different angles. Different people will have different opinions on whether or not, the language is fulfilling the criteria. 