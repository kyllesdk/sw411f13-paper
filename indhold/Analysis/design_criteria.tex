\chapter{Design Criteria}\label{chap:design-criteria}
This section presents the design criteria for the source language. Showing which goals is reached based on the criteria. 

\section{The chosen criteria}
In this design criteria for programming language there will be 5 main criteria, which each have sub-criteria. The criteria are as follows:
\begin{itemize}
\item Readability
\begin{itemize}
\item Overall simplicity
\item Orthogonality
\item Data types and control statements
\item Syntax considerations
\end{itemize}
\item Write-ability
\begin{itemize}
\item Simplicity and Orthogonality
\item Support for abstraction
\item Expressivity
\end{itemize}
\item Reliability
\begin{itemize}
\item Type checking
\item Exception handling
\item Aliasing
\item Readability and write-ability
\end{itemize}
\item Cost
\begin{itemize}
\item Training programmers to use the language
\item Writing programs
\item Compiling programs
\item Executing programs
\item Language implementation system
\item Reliability
\item Maintaining programs
\end{itemize}
\item Others
\begin{itemize}
\item Portability
\item Generality
\item Well-definedness
\end{itemize}
\end{itemize}

However the source language does not fulfil all criteria, and not all of the criteria is needed. As for this language the ``Cost'' criteria as well as the ``Others'' criteria including both their sub-criteria are not needed here. However ``Readability'', ``Write-ability'' and ``Reliability'' are, including their sub-criteria. So the focus will be on these 3 and their respective sub-criteria. \\ \\
\textbf{Readability:} \\
The readability criteria is one of the most important criteria for this language, because this language is developed so that it should be easy to read and thereby easy to understand what it is going on. 
