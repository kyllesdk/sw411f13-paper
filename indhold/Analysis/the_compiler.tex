\chapter{The compiler} \label{chap:the_compiler}
A compiler is a program which translates one language (source language) into another language (target language). When writing a program, the programmer uses a programming language such as Java or C. A computer however only "understands" the language of binary code, the so called machine code. Binary code is in this case the target language reached by translating the source language into the target language, with the use of a compiler. Writing the program in binary code is however not optimal for the programmer as the alphabet of binary consists of 0's and 1's, and can be hard to understand. Writing a program in a high level language such as Java or C and compiling it into machine code is easier for the programmer.
The goal of this project is to translate a high-level language to another high-level language - namely, from the source language to the Arduino language. The task of translating a programming language into another can be done using a compiler. The compiler allows the programmer to avoid using the target language. The compiler is therefore responsible for always producing a correct representation of source language in the target language.

\section{The structure of a compiler} 
\label{sec:compiler}
Compiling a source language into a target language is not a one step process. The compiler goes through several steps in order to translate a program written in a source language into a program written in the target language. Some steps are necessary, and others are optional.

\begin{itemize}
	\item The first task performed by a compiler is scanning. The scanner is part of the lexical analysis. It scans the source language and prepares the code for further treatment. This is done by scanning through the entire code one character at a time. Any unneccessary code, such as the comments and white space, is removed in this process. Every other part of the code is placed in a stream of tokens. These can for example be integers, identifiers and operators. Any user defined elements is furthermore put into a symbol tree, which is accessible throughout the process of compilation. If the user has defined, for example, that whenever the word ``POWER'' is written in the code of the program, it always refers to the number ``9001'', this would be put into the symbol tree. The result is that whenever ``POWER'' is found in the code, it will be replaced with ``9001''. After the source program has been run through, the output from the scanner is passed on to the parser.
	
	\item The second task is parsing. A parser analyses the syntax of the code received from the scanner. The parser checks if the syntax of the code is compatible to the syntax of the target language. If the written code is not proper according to the target language, it is not possible to be translated correctly. As an example, a C parser will return an error if the code states $a + b = c;$ rather than $c = a + b;$ this is because the syntax of the first expression is illegal in C. It is important to note that at this step it is unknown what the variables contain, which is where the type checker takes over.

	\item After the syntax has been verified, the third step is the type checking, which deals with the semantics of the code. The type checker verifies that all operations performed in the code are legal regarding to the language. In the previous example, $c = a + b;$ is syntactically correct. However what has not been specified is the content of a, b and c. If a and b are strings (text), while c is of the type int (number), the semantic is incorrect and will therefore return an error as the operation does not exist in the defined source language. If no errors occur, the parser will return an abstract syntax tree (AST) which also concludes the analysis of the source code. As seen in figure \ref{fig:ast_example} the AST shows the syntactic constructs. For example, an assignment requires a variable and an expression. 
	
	\begin{figure}[H]
	\centering
		\includegraphics{billeder/example_ast.png}
		\caption{AST example}
		\label{fig:ast_example}
\end{figure}
	
	

	\item At this point, the preparation of the source code is complete and the translation into the target language can be performed correctly. This is done by the translator. The translator translates the generated AST, from the parser, to intermediate representation code. IR code is used to describe the meaning of the AST nodes. For example to describe that a while loop actually loops. There is nothing in the AST that implies this.     
In compilers that do not implement optimization, target code can be generated by a translator without the use of an explicit IR.

To translate the AST code generation visitors (CodeGenVisitor) are used, which visit the nodes in the AST. There are various different CodeGenVisitors to generate the right low-level language code. The output from the translator is the compiled product of the source code. \todo{Matti: Revise this paragraph about CodeGenVisitor} 
\end{itemize}

While this covers the direct compiling process, more optional steps can be used to enhance the compiling process for different purposes.\\

\begin{itemize}
\item An optimizer can be used to analyze, and generate more effective IR code. While a piece of code can be optimal in the source language, the target language might have a more efficient formulation for describing the same action than the direct translation produces. The optimizer locates code which can be optimized and replaces the code with the more efficient model. 

\item In order to generate target machine code, a code generator can be used to translate the IR code generated by the translator.  \todo{Matti: If anyone has the balls for it, please elaborate about code generator / code generation. Checkc the Crafting a Compiler book for more information}

\end{itemize}
