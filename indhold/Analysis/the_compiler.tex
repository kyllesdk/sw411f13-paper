\section{The general compiler}
A compiler is a program which compiles one language into another. When writing a program to run on a PC, the programmer writes a program in a source language such as Java or C. The PC however only understand the language of binary, also called machine code. Binary is in this case the target language accomplished by translating the source language into the target language. Writing the program in binary code is however not optimal for the programmer as the alphabet of binary consisting of 0 and 1 is hard to keep track of. There are no abstractions as opposed to higher level (abstract) languages where for example sensible names for variables can be used for easier reading. Specific for this project the goal is to translate a high-level language to another high-level language - Namely from our language to Wiring. The task of translating a programming language to another can be done with a compiler which is basically a translator. The compiler frees the programmer of having to translate the program or write entirely in the target language. The compiler is therefore responsible for always producing a correct representation of source language in the target language.

\subsection{The structure of a compiler}
The task of compiling a source language into a target language is not a one step process. The compiler goes through several steps to produce the target language. Some steps are optional to the task of compiling one language into another.\cite{compiler:structure}\\

\begin{itemize}
	\item The first step a compiler performs is the scanner which prepares the source code\todo{Morten: Syntax and semantics should be explained before this}. for further treatment. This is done by running through the entire program one character at a time. Although optional for most translations, it is common\todo{Morten: Need source} source code are always removed.  Every other part of the code is placed in groupings (int, float, ;, =, etc). Any user defined elements is furthermore put into a symbol tree which is accessible throughout the process of compilation. If the user for example defined POWER to always be the number 9001, this would be put into the symbol tree. The result is that whenever POWER is found in the code, it can be replaced with 9001. After the source program has been run through, the output from the scanner is passed on to the parser.
	
	\item The parser analyzes the syntax of the code passed on by the scanner. The parser identifies that everything is in order according to the defined syntax of the source language. If the written code is not proper according to the source language it would not be possible to translate it correctly. As example, a C parser will return an error if the code states $a + b = c;$ rather than $c = a + b;$ because the syntax is incorrect according to the syntax definition of C.

	\item After the syntax has been verified the type checker deals with the semantics of the source code. The type checker ensures that all operations performed in the code is legal with regard to the source language. To continue the previous example, $c = a + b;$ is syntactically correct. But what has not been mentioned is the content of a, b and c. If a and b are strings, while c is of the type int, the semantics are incorrect and will therefore return an error as the operation does not make sense in the source language.

	\item The preparation of the source code is now complete and the translation into the target language can be performed correctly. This is done by a translator\todo{Morten: Will be elaborated on as the course progresses}. The output from the translator is the compiled product of the source code.
\end{itemize}

While this covers the direct compiling process, more optional steps can be used to enhance the compiling process for different purposes.\\

\begin{itemize}
\item An optimizer can be used produce more effective code in the target language. While a piece of code can be optimal in the source language, the target language might have a more efficient formulation for describing the same action than the direct translation would produce. The optimizer locates code which can be optimized and replaces the code with the more efficient model.

\item...\todo{There are probably more steps to be added (IR?)}
\end{itemize}
