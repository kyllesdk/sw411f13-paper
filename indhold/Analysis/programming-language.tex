\chapter{Source language}\label{analysis:source-language}
In this project a new programming language\todo{name of language} is proposed. The language is designed for the use of beginners in the field of programming and it is meant  to be an alternative for the already existing Arduino programming language.
The goal of the new language, the source language, is to make programming simpler and easier to understand compared to the existing Arduino programming language. The syntax is meant to be intuitive in that it can be read naturally, similar to the English language.

The source language is compiled to the Arduino language, rather than machine code. The source language is not a fully functioning programming language with advanced features. However, it is a simple language that can print out text and handle simple mathematics, such as adding, subtracting, multiplying numbers etc. along with handling boolean expressions built up from equalities and inequalities of numeric expressions.

The features for the source language are listed below:
\begin{itemize}
	\item It is readable, also by users that are not familiar with programming languages.
	\item It is easy to learn, and to be used for basic programming.
	\item It can be compiled to Arduino.
	\item The language is able to handle mathematics and boolean expressions.
\end{itemize}