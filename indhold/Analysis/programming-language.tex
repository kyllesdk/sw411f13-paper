\chapter{Source language}\label{analysis:source-language}
In this project a new programming language is proposed. The language is designed for the use of beginners in the field of programming and is meant as an alternative for the already existing Arduino programming language (See chapter \ref{analysis:arduino}).
The goal of the new language, the source language, is to make it simpler and easier to understand compared to the existing Arduino programming language. The syntax should be intuitive in that it can be read naturally, similar to the English language.

The source language is compiled to the Arduino language, rather than machine code. The source language is not a fully functioning programming language with advanced features. However, it is a simple language that can print out text and handle simple mathematics, such as adding, subtracting, multiplying numbers etc. along with handling boolean expressions, built up from equalities and inequalities of numeric expressions.

The features for the source language are listed below:
\begin{itemize}
	\item Readable, also by users that are not familiar with programming languages.
	\item Easy to learn, and to be used for basic programming.
	\item Should be compiled to Arduino.
	\item The language should be able to handle mathematics and boolean expressions.
\end{itemize}

\subsection{Python}
The source language that is developed in this project has a syntax that is somewhat similar to Python. Python is a programming language that is open source and supported by operating systems like: Windows, Linux/Unix, OSX. Python is a language with a syntax similar to English. It has been developed following some of the same criteria as our source language. Readable syntax is one of the main features, and it is the reason for creating the source language based on Python.
 
\cite{python:official:about}
There exists a large community of users of Python, which help each others with different problems. There are also many tutorials on the internet and books that teach how to use the language. Python's official website has a tutorial guide.\ref{python:about} \\
Python was created back in the 1980s. Its first official release was in 1994. The language was created by Guido van Rossum. When he created Python, the language was also influenced from other existing languages, mainly ABC, which is the core syntax to Python. \ref{python:wiki}