\chapter{Source language}\label{analysis:source-language}
In this project a new programming language has been made. The language is aimed at people who are beginners in the field of programming. This programming language is a replacement for the already existing Arduino programming language chapter \ref{analysis:arduino}.
The goal of the new programming language, the source language, is for it to be simpler and easier to understand than the existing Arduino programming language. The syntax should be readable, and close to English.

The source language is compiled to the Arduino languag, and the source language will not be compiled to machine code. The source language is not a fully functioning programming language with advanced features. However, it is a simple language that can print out text and handle simple mathematics, such as adding, subtracting, multiplying etc. along with boolean expressions, such as greater than, less than, equal to.

The features for the source language are listed below:
\begin{itemize}
	\item Readable, also by users that are not familiar with programming languages.
	\item Easy to learn, and to be used for doing basic programming.
	\item Should be compiled to Arduino.
	\item The language should be able to handle mathematics and boolean expressions.
\end{itemize}

\subsection{Python}
The source language that is developed have syntax that is somewhat similar to Python. Python is a programming language that is open source and supported by operating systems like: Windows, Linux/Unix, OSX.

Python is a language with a syntax similar to English.
Python has been developed with some of the same criteria as our source language. Readable syntax is one of the keywords, and is the reason for creating the source language based on Python.
 
\cite{python:official:about}
There exists a large community of users which uses Python, they help each others with different problems. There are also many tutorials on the internet and in books that teach how to use the language. Python's official website even has a tutorial guide.\cite{python:official:tutorial}\todo{Jonas: Vi skal nok skrive noget mere on python}.