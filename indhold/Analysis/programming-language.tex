\section{Source language}\label{analysis:source-language}
In this project a new programming language is supposed to be made. The language is supposed to be used be people that are beginners in the field of programming. The programming language is a replacement for the already existing Arduino programming language.\todo{Jonas: Vi skal nok have et afsnit om arduino vi kan henvise til} 
The goal with the new language is that it should be easier to implement than the existing Arduino programming language. The syntax should be more readable. The syntax should be closer to a natural language like English than the already existing language is.

The language should be compiled down to the Arduino language. The language will not be compiled to machine code. The source language will not be a fully functioning programming language with advance features instead it will be a simple language that can print out text and handle simple mathematics and boolean expressions.

The requirements for the language is listed below:
\begin{itemize}
	\item Very readable, also by users that are not familiar with programming languages.
	\item Easy to learn how to do basic programming in the language.
	\item Should be compiled down to Arduino.
	\item The language should be able to handle mathematics and boolean expressions.
\end{itemize}

\subsection{Python}
The programming language that will be developed should have syntax that is similar to Python. Python is a language that is open source and is supported by operating systems like: Windows, Linux/Unix, OSX.

Python is a language which syntax look like written English instead of a programming language.
Python focus on things like:

\begin{itemize}
	\item Readable syntax.
	\item Intuitive object oriented.
	\item Dynamic data types.
	\item Embeddable within applications as a scripting interface.
\end{itemize} 
\cite{python:official:about}
Python has a large community where users of Python helps each other with different problems. There are also many tutorials on the internet and in books that teach how to use Python. Python's official site even have a tutorial site.\cite{python:official:tutorial}\todo{Jonas: Vi skal nok skrive noget mere on python}.