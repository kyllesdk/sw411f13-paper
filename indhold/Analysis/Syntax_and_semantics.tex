\section{Syntax and Semantics}
In this section a small description of the meaning of syntax and semantics are described.

\subsection{Syntax}
When describing a programming language the syntax of the language is the set of rules which define the character combinations allowed when writing a program in that language. To define the syntax of a language regular expressions are being used, which specifies what strings are recognized in the language. The syntax of the language is also described via Backus-Naur Form, short BNF, that describes the context free grammar of the language with nonterminal and terminal symbols.
EXAMPLE: \\
\begin{lstlisting}
<program> -> begin <stmt_list> end
<stmt_list> -> <stmt>
			| <stmt> ; <stmt_list>
<stmt> -> <var> = <expression>
<var> -> A | B | C
<expression> -> <var> + <var>
			| <var> - <var>
			| <var>
\end{lstlisting}

\textbf{IF-statement}
\begin{lstlisting}
<if_stmt> -> if(<expression>) then <statement> end
				| if(<expr>) then <stmt> elseif <stmt> else <stmt> end
				| if(<expr>) then <statement> else <statement> end
\end{lstlisting}

\textbf{While loop}
\begin{lstlisting}
<while_stmt> -> while (<expr>) do <stmt> end
\end{lstlisting}

\texfbf{Function}
\begin{lstlisting}
<function> -> function (<expr>) do <stmt> end
\end{lstlisting}

\subsection{Semantics}
The semantics of a language is the description of what happens when a program is executed.