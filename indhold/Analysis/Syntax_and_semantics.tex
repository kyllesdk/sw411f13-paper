%\section{Syntax}
%When describing a programming language the syntax of the language is given by the set of rules which defines the words (finite sequences of characters) that can be used for writing a program in that language. The syntax of the language is described using Backus-Naur Form (BNF) which provides the context free grammar of the language.
%In this report an extended version of BNF, Extended Backus-Naur Form (EBNF), is used. The advantage of using EBNF is to describe the set of rules in a more compact form, and the ability to describe regular expressions in the context free grammar. EBNF does not enhance the descriptive power of BNF, it only increases the readability and the write-ability.

%\subsection{Program}
% \begin{lstlisting}[captionpos=b, caption{EBNF of how a program needs to be structured.}]
% <program> -> void setup() begin <statement_list> end void loop() begin <statement_list> end
% <statement_list> -> (<statement> | <statement> ; <statement_list>
% \end{lstlisting}

% \subsection{IF-statement}
% The following EBNF describes how IF-statements can be used in the source language. It describes that the if-statement can be used in one of the three different ways:
% \begin{itemize}
% 	\item if(EXPRESSION) then STATEMENT end
% 	\item if(EXPRESSION) then STATEMENT elseif STATEMENT else STATEMENT end. The elseif statement can be used an infinite amount of times.
% 	\item If(EXPRESSION) then STATEMENT else STATEMENT end
% \end{itemize}
% \begin{lstlisting}[captionpos=b, caption={EBNF of the IF-statement in the source language}]
% <if_statement> -> if(<expression>) then <statement> [{elseif <statement>} else <statement> | else <statement>] end
% \end{lstlisting}


% \subsection{While loop}
% The following EBNF describes how the while loop is allowed to be used in the source language. Note that only a while loop is described, because no other loops than the while loop is part of the source language.
% \begin{lstlisting}[mathescape, captionpos=b, caption={EBNF of a while loop.}]
% <while_stmt> -> while (<expression>) do <statement> end
% \end{lstlisting}

% \subsection{Function}
% \begin{lstlisting}[mathescape, captionpos=b, caption={EBNF of a function.}]
% <function> -> function (<expression>) do <statement> end
% \end{lstlisting}

% \subsection{Assign}
% \begin{lstlisting}[mathescape, captionpos=b, caption{EBNF of how assignment work.}]
% <assign> -> <variable> = <expression>;
% \end{lstlisting}

% \subsection{Variables}
% \begin{lstlisting}[mathescape, captionpos=b, caption{EBNF of how a variable can be declared.}]
% <variable> -> (<arithmetic expression> | $\lambda$);
% \end{lstlisting}

% \subsection{Expression}
% \begin{lstlisting}[mathescape, captionpos=b, caption{EBNF of expressions.}]
% <arithmetic expression> -> <variable> (+ | - | *) <variable>;
% \end{lstlisting}
\newpage
\section{Semantics} \label{sec:semantics}
The semantics of a language is the description of what happens when a program is executed. In table \ref{tab:semantics_aritmethic}, \ref{tab:semantics_boolean} and \ref{tab:semantics_statements} are the descriptions of the operations in the source language. \\

In table \ref{tab:semantics_aritmethic} the arrow $\rightarrow_{a}$ is used. This arrow represents a transition. For instance, the first line in ``PLUS'', represents the transition from state $a_1$ to $v_1$ as well as the transition from $a_2$ to $v_2$. The label on the arrow indicates that the transition system only allows values that all are arithmetic expressions.

\begin{table}[H]
	\begin{tabular}{|l|l|l|}
	\hline
	\emph{Name}			& \emph{Rule}																															& \emph{Notes} \\ \hline
			~			&															~																			& ~ \\
	$[PLUS_{BSS}]$		& $env_{v},~sto~\vdash~a_{1}~\rightarrow_{a}~v_{1}~~ env_{v},~sto~\vdash~a_{2}~\rightarrow_{a}~v_{2} \over env_{v},~sto~\vdash~a_{1}~+~a_{2}~\rightarrow_{a}~v$ 			& where $v = v_{1}+v_{2}$ \\
			~			&															~																			& ~ \\
	$[MINUS_{BSS}]$		& $env_{v},~sto~\vdash~a_{1}~\rightarrow_{a}~v_{1}~~ env_{v},~sto~\vdash a_{2}~\rightarrow_{a}~v_{2} \over env_{v},~sto~\vdash~a_{1}~-~a_{2}~\rightarrow_{a}~v$ 			& where $v = v_{1}-v_{2}$ \\
			~			&															~																			& ~ \\
	$[MULT_{BSS}]$		& $env_{v},~sto~\vdash~a_{1}~\rightarrow_{a}~v_{1}~~env_{v},~sto~\vdash a_{2}~\rightarrow_{a}~v_{2} \over env_{v},~sto~\vdash~a_{1}~*~a_{2}~\rightarrow_{a}~v$ 			& where $v = v_{1}*v_{2}$ \\
			~			&															~																			& ~ \\
	$[DIV_{BSS}]$		& $env_{v},~sto~\vdash a_{1}~\rightarrow_{a}~v_{1}~~env_{v},~sto ~\vdash a_{2}~\rightarrow_{a}~v_{2} \over env_{v},~sto~\vdash~{a_{1}}~/~{a_{2}}~\rightarrow_{a}~v$ 		& where $v = \frac{v_{1}}{v_{2}}$ and $v_2 \neq 0$\\
			~			&																																		& ~ \\
	$[MOD_{BSS}]$		& $env_{v},~sto~\vdash a_{1}~\rightarrow_{a}~v_{1}~~env_{v},~sto~\vdash~a_{2}~\rightarrow_{a}~v_{2} \over env_{v},~sto~\vdash~a_{1}~\%~a_{2}~\rightarrow_{a}~v$			& where $v = v_{1} mod v_{2}$ \\
			~			&															~																			& ~ \\
	$[POW_{BSS}]$		& $env_{v},~sto~\vdash~a_{1}~\rightarrow_{a}~v_{1}~~env_{v},~sto~\vdash~a_{2}~
	\rightarrow_{a}~v_{2} \over~env_{v},~sto~\vdash~a_{1}~\texttt{\^{}}~a_{2}~\rightarrow_{a}~v$	& where $v = v_{1}\texttt{\^{}}v_{2}$ \\
			~			&															~																			& ~ \\
	$[SQRT_{BSS}]$		& $env_{v},~sto~\vdash~a_{1}~\rightarrow_{a}~v_{1} \over env_{v},~sto~\vdash~sqrt(a_{1})~\rightarrow_{a}~v_{1}$												& where $v = \sqrt{v_{1}}$ and $v_1 >= 0$ \\
			~			&															~																			& ~ \\
	$[PARENT_{BSS}]$	& $env_{v},~sto~\vdash a_{1}~\rightarrow_{a}~v_{1} \over env_{v},~sto~\vdash~(a_{1})~\rightarrow_{a}~v_{1}$													& ~ \\
			~			&															~																			& ~ \\
	$[NUM_{BSS}]$		& $env_{v},~sto~\vdash~n~\rightarrow_{a}~v~ if ~\mathcal{N}\llbracket n \rrbracket~=~v$															& ~ \\
			~			&															~																			& ~ \\
	$[VAR_{BSS}]$		& $env_{v},~sto~\vdash~x~\rightarrow_{a}~v~if~sx~=~v$														   									& ~ \\
			~			&															~																			& ~ \\
	\hline
	\end{tabular}
	\label{tab:semantics_aritmethic}
	\caption{The semantics of arithmetic operations}
\end{table}

First is table \ref{tab:semantics_aritmethic}, which describes the arithmetic expressions - the more common ones are addition, subtraction, multiplication and division (PLUS(+), MINUS(-), MULT(*), DIV(/)). Their rules all look near identical: Within a variable environment $env_{v}$ with a store $sto$, two expressions, $v_{1}$ and $v_{2}$, which both are arithmetic as seen by $\rightarrow_{a}$ can be added, subtracted, multiplied or divided. This results in $v$, which is a value. However, when taking a look upon DIV's side-condition, take notice that it states that $v_{2}$ cannot be divided by zero, as it is not possible to divide any number by zero, so this side-note makes sure that it is not possible to do so. \\
Furthermore, $v$ is the product of the operator used on $v_{1}$ and $v_{2}$. Both modulus and power of (MOD(\%), POW($\texttt{\^{}}$) are identical to these operations in procedure. \\
SQRT, also known as square-root($\sqrt{}$) is a simple rule to allow any positive arithmetic expression, $v_{1}$ to be squared, where $v$ is $\sqrt{v_{1}}$. Like on the transition rule for division, there is an extra side condition, which indicates that the number that is to be square-rooted is either a zero or a positive number, as the source language does not going to allow imaginary numbers. \\

The \textbf{PARENT}, or parentheses(()), rule states that any arithmetic expression can be parenthesized, allowing chained arithmetic calculations to be performed in the correct order. \\\textbf{VAR} describes, that within a variable enviroment $env_{v}$ with a store $sto$, a value $v$ can be assigned to a variable $x$, allowing variable declaration.
\\\textbf{NUM} or numeral also known as absolute values, allows within an environment to get the numerical value of a variable $v$. A numeric value is the distance a number has from zero, for instance $|2|$ and $|-2|$ are both 2 units from 0.  \\

\begin{table}[H]
	\begin{tabular}{|l|l|l|}
	\hline
	\emph{Name}			& \emph{Rule}																															& \emph{Notes} \\ \hline
			~			&															~																			& ~ \\
	$[NOT1_{BSS}]$		& $s~\vdash~b_{1}~\rightarrow_{b}~tt \over s~\vdash~not~b_{1}~\rightarrow_{b}~ff$														& ~ \\
			~			&															~																			& ~ \\
	$[NOT2_{BSS}]$		& $s~\vdash~b_{1}~\rightarrow_{b}~ff \over s~\vdash~not~b_{1}~\rightarrow_{b}~tt$														& ~ \\
			~			&															~																			& ~ \\
	$[EQUAL1_{BSS}]$	& $s~\vdash~a_{1}~\rightarrow_{a}~v_{1}~~s~\vdash a_{2}~\rightarrow_{a}~v_{2} \over s~\vdash~a_{1}~equals~a_{2}~\rightarrow_{b}~tt$		& if $v_{1} = v_{2}$ \\
			~			&															~																			& ~ \\
	$[EQUAL2_{BSS}]$	& $s~\vdash~a_{1}~\rightarrow_{a}~v_{1}~~s~\vdash~a_{2}~
	\rightarrow_{a}~v_{2} \over s~\vdash~a_{1}~equals~a_{2}~\rightarrow_{b}~ff$		& if $v_{1} \ne v_{2}$ \\
			~			&															~																			& ~ \\
	$[NOTEQUAL1_{BSS}]$	& $s~\vdash~a_{1}~\rightarrow_{a}~v_{1}~~s~\vdash~a_{2}~
	\rightarrow_{a}~v_{2} \over s~\vdash~a_{1}~notequals~a_{2}~\rightarrow_{b}~tt$	& if $v_{1}\ne v_{2}$ \\
			~			&															~																			& ~ \\
	$[NOTEQUAL2_{BSS}]$	& $s~\vdash~a_{1}~\rightarrow_{a}~v_{1}~~s~
	\vdash~a_{2}~\rightarrow_{a}~v_{2} \over s~\vdash~a_{1}~notequals~a_{2}~\rightarrow_{b}~ff$	& if $v_{1} = v_{2}$ \\
			~			&															~																			& ~ \\
	$[GREATER1_{BSS}]$	& $s~\vdash~a_{1}~\rightarrow_{a}~v_{1}~~s~
	\vdash~a_{2}~\rightarrow_{a}~v_{2} \over s~\vdash~a_{1}~<~a_{2}~\rightarrow_{b}~tt$			& if $v_{1} < v_{2}$ \\
			~			&															~																			& ~ \\
	$[GREATER2_{BSS}]$	& $s~\vdash~a_{1}~\rightarrow_{a}~v_{1}~~s~
	\vdash~a_{2}~\rightarrow_{a}~v_{2} \over s~\vdash~a_{1}~<~a_{2}~\rightarrow_{b}~ff$			& if $v_{1} \not< v_{2}$ \\
			~			&															~																			& ~ \\
	$[LESSER1_{BSS}]$	& $s~\vdash~a_{1}~\rightarrow_{a}~v_{1}~~s~\vdash~a_{2}~
	\rightarrow_{a}~v_{2} \over s~\vdash~a_{1}~>~a_{2}~\rightarrow_{b}~tt$			& if $v_{1} > v_{2}$ \\
			~			&															~																			& ~ \\
	$[LESSER2_{BSS}]$	& $s~\vdash~a_{1}~\rightarrow_{a}~v_{1}~~s~\vdash a_{2}~\rightarrow_{a}~v_{2} \over s~\vdash~a_{1}~>~a_{2}~\rightarrow_{b}~ff$			& if $v_{1} \not> v_{2}$ \\
			~			&															~																			& ~ \\
	$[LEQ1_{BSS}]$		& $s~\vdash~a_{1}~\rightarrow_{a}~v_{1}~~s~\vdash~a_{2}~
	\rightarrow_{a}~v_{2} \over s~\vdash~a_{1}~>=~a_{2}~\rightarrow_{b}~tt$			& if $v_{1} \leq v_{2}$ \\
			~			&															~																			& ~ \\
	$[LEQ2_{BSS}]$		& $s~\vdash~a_{1}~\rightarrow_{a}~v_{1}~~s~\vdash~a_{2}~
	\rightarrow_{a}~v_{2} \over s~\vdash~a_{1}~>=~a_{2}~\rightarrow_{b}~ff$			& if $v_{1} \not\leq v_{2}$ \\
			~			&															~																			& ~ \\
	$[GEQ1_{BSS}]$		& $s~\vdash~a_{1}~\rightarrow_{a}~v_{1}~~s~\vdash~a_{2}~
	\rightarrow_{a}~v_{2} \over s~\vdash~a_{1}~<=~a_{2}~\rightarrow_{b}~tt$			& if $v_{1} \geq v_{2}$ \\
			~			&															~																			& ~ \\
	$[GEQ2_{BSS}]$		& $s~\vdash~a_{1}~\rightarrow_{a}~v_{1}~~s~\vdash~a_{2}~
	\rightarrow_{a}~v_{2} \over s~\vdash~a_{1}~<=~a_{2}~\rightarrow_{b}~ff$			& if $v_{1} \not\geq v_{2}$ \\
			~			&															~																			& ~ \\
	$[PARENT_{BSS}]$	& $s~\vdash~b_{1}~\rightarrow_{b}~v_{1} \over s~\vdash~(b_{1})~\rightarrow_{b}~v_{1}$													& ~ \\
			~			&															~																			& ~ \\
	$[AND1_{BSS}]$		& $s~\vdash~b_{1}~\rightarrow_{b}~tt~~s~\vdash~b_{2}~
	\rightarrow_{b}~tt \over s~\vdash~b_{1}~and~b_{2}~\rightarrow_{b}~tt$				& ~ \\
			~			&															~																			& ~ \\
	$[AND2_{BSS}]$		& $s~\vdash~b_{i}~\rightarrow_{b}~ff \over s~\vdash~b_{1}~and~b_{2}~\rightarrow_{b}~ff$													& where $i \in \{1,2\}$ \\
			~			&															~																			& ~ \\
	$[OR1_{BSS}]$		& $s~\vdash~b_{1}~\rightarrow_{b}~tt~~s~\vdash~b_{2}~
	\rightarrow_{b}~tt \over s~\vdash~b_{1}~or~b_{2}~\rightarrow_{b}~tt$				& ~ \\
			~			&															~																			& ~ \\
	$[OR2_{BSS}]$		& $s~\vdash~b_{i}~\rightarrow_{b}~ff \over s~\vdash~b_{1}~or~b_{2}~\rightarrow_{b}~ff$													& ~ \\
			~			&															~																			& ~ \\
	\hline
	\end{tabular}
	\caption{The semantics of boolean operations}
	\label{tab:semantics_boolean}
\end{table}

Table \ref{tab:semantics_boolean} holds the rules of boolean operations. Once again a lot of operations look near identical, namely EQUAL1(=), EQUAL2($\neq$), NOTEQUAL1($\neq$), NOTEQUAL2(=), GREATER1(<), GREATER2($\not<$), LESSER1(>), LESSER2($\not>$), LEQ1($\leq$), LEQ2($\not\leq$), GEQ1($\geq$) and GEQ2($\not\geq$). The common factor is that these are all comparison expressions, respectively checks for equality, equals, not equals, greater than, less than, less or equal to and greater or equal to. Having a $v_{1}$ and $v_{2}$, which are both arithmetic as seen by $\rightarrow_{a}$, a boolean evaluation is used to conclude whether comparison is true (tt) or false (ff). \\
The NOT1($\neg$), NOT2($\neg$), AND1($\wedge$), AND2($\wedge$), OR1($\vee$) and OR2($\vee$) expressions are similar as well. Arithmetic expressions are tested with some criteria and evaluated to true or false. In the case of NOT1 where the evaluation is true, the evaluation of NOT1 is obviously false as it is not ``not'', with NOT2 being the opposite case. AND1, AND2, OR1 and OR2 follow same principles. In AND1 and AND2, two expressions need to be true if the entire expression is to be true, while OR1 and OR2 just needs one of two expressions. \\
As with the PARENT rule in table \ref{tab:semantics_aritmethic} any boolean expression can be parenthesized as well as seen in this PARENT rule. \\ 

\begin{table}[H]
	\begin{tabular}{|l|l|l|}
	\hline
	\emph{Name}			& \emph{Rule}																															& \emph{Notes} \\ \hline
			~			&															~																			& ~ \\
	$[ASS_{BSS}]$		& $\langle x~:=~a,s \rangle \rightarrow s[x \mapsto v]$																					& where $s \vdash a \rightarrow_{a} v$ \\
			~			&															~																			& ~ \\
	$[SKIP_{BSS}]$		& $\langle skip, s \rangle \rightarrow s$																								& ~ \\
			~			&															~																			& ~ \\
	$[COMP_{BSS}]$		& $\langle S_{1},s \rangle~\rightarrow~s''~~\langle S_{1},s'' \rangle~\rightarrow~s'  \over \langle S_{1};S_{2},s \rangle~\rightarrow~s'$	& ~ \\
			~			&															~																			& ~ \\
	$[IFTRUE_{BSS}]$	& $\langle S_{1},s \rangle~\rightarrow~s' \over \langle if~b~do~S_{1}~else~S_{2},s \rangle~\rightarrow~s'	$							& if $s \vdash b \rightarrow_{b} tt$\\
			~			&															~																			& ~ \\
	$[IFFALSE_{BSS}]$	& $\langle S_{2},s \rangle~\rightarrow~s' \over \langle if~b~do~S_{1}~else~S_{2},s \rangle~\rightarrow~s'	$							& if $s \vdash b \rightarrow_{b} ff$\\
			~			&															~																			& ~ \\
	$[WHILETRUE_{BSS}]$	& $\langle S,s \rangle~\rightarrow~s'' \langle while~b~do~S,s'' \rangle~\rightarrow~s' \over \langle while~b~do~S,s \rangle~\rightarrow~s'	$ & if $s \vdash b \rightarrow_{b} tt$\\
			~			&															~																			& ~ \\
	$[WHILEFALSE_{BSS}]$& $\langle while~b~do~S,s \rangle~\rightarrow~s~if~s~\vdash~b~\rightarrow_{b}~ff $														& ~ \\
	\hline
	\end{tabular}
	\label{tab:semantics_statements}
	\caption{The semantics of statements}
\end{table}

The table of statement semantics, \ref{tab:semantics_statements} holds rules affecting statements. \\
ASS, assignments, describes that with a given expression and state ($a, s$), a variable $x$ can be assigned a value $v$. The SKIP rule states that within a state $s$, a skip can occur, leading to a state $s$. COMP, composition, allows compositional statements meaning that a series of statements can occur. IFTRUE and IFFALSE allows choosing an appropriate sub-statement based on boolean expression within a statement. WHILETRUE allows the statement to call itself again, making it recursive for as long as the expression is true. Whereas WHILEFALSE will do nothing for as long as the expression is false.\\ 
All of these rules, arithmetic, boolean and statements makes the base of the source language.

