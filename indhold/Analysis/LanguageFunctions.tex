\section{Language functions}
A basic classification of functions of the language is presented below.
\begin{itemize}
\item \textbf{Must have}: The features that guarantee the minimum requirements for a functional language, as demanded by the purpose of this project.

\item \textbf{Should have}: Features that are important to the project but does not impair the core functionality of the project.

\item \textbf{Could have}: Features that would be nice to have, but are not of greater importance to accomplish for this project.
\end{itemize}
This overview is based upon the MoSCoW method, which is a technique used in business analysis as well as in software development. It is used to set the importance level of the various criteria needed for the project. \ref{moscow}. However in this MoSCoW analysis the ``Would like'' part is left out, as it has no relevance. 

For this project the following have been chosen:\\

\textbf{Must have:}
\begin{itemize}
\item If-statement \\
The feature of choosing one path over another is so common to programming that it would make no sense to leave it out. One must be able to choose what to do based on something important to the program. \\

\item Expressions \\
In relation to if-statements it is impossible to leave out the ability to evaluate expressions to true or false. \\

\item Functions, parameters and return values \\
The ability to reuse code can reduce the cluttering of code significantly. While it may be possible to leave out functions, it will not make sense from a user-friendly perspective. Making use of parameters and return values removes the complexity of passing by reference. \\

\item Loop - While \\
Another very common task is to repeat the same task multiple times. Instead of having to create the code multiple times it is more efficient to simply run the same piece of code. \\

\item Blocks (do/end signs) \\
Making it clear for the user where a block begins and ends helps reduce confusion. For example, it might be hard to figure where a while loop ends if it has no ``end'' keyword or other recognizable sign. \\

\item Print \\
The ability to print out the results of a program is an easy way for the programmer to see the outcome. \\

\item Logical operators \\
Being able to select and group operators is important to perform decision making in programs. It would be very difficult to leave out of a general purpose language.\\

\item Simple mathematical operators \\
Leaving out mathematics removes a lot of functionality and heavily reduces the options of programming. \\

\item Type definitions \\
The user has to be capable of identifying what is being used. Without clearly defined types to use it can be very difficult for a beginner to understand how to program. \\

\item Imperative language \\
Choosing Imperative language means that the user will have to write code, that describes in exact details what steps has to be done and what is the next step. The Imperative language are supported by most of the mainstream object-oriented programming languages such as Java, C\#. \ref{Impr}

\item UTF-8 \\
UTF-8 is short for \textbf{U}CS(Universal Character Set) \textbf{T}ransformation \textbf{F}ormat-8bit. It can represent every character in the Unicode set and provides compatibility with ASCII. 
The reason UTF-8 is a must have is because the typeset is very broad. This means the user does not have to worry about whether or not his special characters is acceptable characters, and does as such give the user the ability to code naturally.\\
\end{itemize}

\textbf{Should have:}
\begin{itemize}
\item Initialization \\
There are different ways that initialization can work depending on the language as well as the type of the object. \\
\end{itemize}

\textbf{Could have:}
\begin{itemize}
\item User input \\
While definitely a feature which is relevant for some purposes, the feature is not required in order to have a programming language able to use the features of Arduino. \\

\item Advanced mathematical operators \\
Mathematics can be solved without the use of integrated mathematics such as square root and modulus, but having the features is definitely convenient. \\

\item Loops (Other than while) \\
The ``while-loop'' has all the functionality needed, but in some cases there are more intuitive ways to go around a problem. ``Foreach'' is a good example of this problem.\\
\end{itemize}