\section{Language functions}
To get an overview and a basic representation of importance the language functions has been split into groupings called:
\begin{itemize}
\item \textbf{Must have}: features which can not be left out in order to still have a functional language for the purpose of the project.
\item \textbf{Should have}: features which are important to the project but does not impair the core functionality of the project.
\item \textbf{Could have}: features which would be nice to have, but are not of greater importance to accomplish for this project.
\end{itemize}

\textbf{Could haves:} \\

User input \\
While definitely a feature which is relevant for some purposes, the feature is not required in order to have a programming language able to use the features of Arduino. \\

Advanced mathematical operators \\
Mathematics can be solved without the use of integrated mathematics such as square root and modulus, but having the features is definitely convenient. \\

Loops (Other than while) \\
The ``while'' has all the functionality needed, but in some cases there are more intuitive ways to go around a problem. Foreach is a good example of this problem.\\

\textbf{Should haves:} \\

Initialization \\


\textbf{Must haves:} \\

If-statement \\
The feature of choosing one path over another is so common to programming that it would make no sense to leave it out. One must be able to choose what to do based on something important to the program. \\

Boolean expressions \\
In relation to if-statements it makes little sense to leave out the ability to evaluate expressions to true or false. \\

Functions, parameters and return values \\
The ability to reuse code can reduce the cluttering of code significantly. While it may be possible to leave out functions, it will not make sense from a user-friendly perspective. Making use of parameters and return values removes the complexity of passing by reference. \\

Loop - While \\
Another very common task is to repeat the same task multiple times. Instead of having to create the code multiple times it is more user friendly to simple run the same piece of code, and is as such a must have. \\

Blocks (do/end signs) \\
Making it clear for the user where a block begins and ends helps reduce confusion. For example it might be hard to figure where a while loop ends if it has no ``end'' keyword or other recognizable sign. \\

Print \\
The ability to print out the results of a program is an easy way for the programmer to see the outcome. It is easy and user friendly. \\

Logical operators \\
Being able to select and group operators is important to perform decision making in programs. It would be very difficult to leave out of a general purpose language.\\

Simple mathematical operators \\
Leaving out mathematics removes a lot of functionality and heavily reduces the options of programming. \\

Type definitions \\
The user has to be capable of identifying what is being used. Without clearly defined types to use it can be very difficult for a beginner to understand how to program. \\

Imperative language\todo{Missing argumentation} \\ 

UTF-8 \\
The reason UTF-8 is a must have is because the typeset is very broad. This means the user does not have to worry about whether or not his special characters is acceptable characters, and does as such give the user the ability to code naturally.