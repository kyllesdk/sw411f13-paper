\section{MoSCoW analysis}\label{analysis:moscow}
This subsection is based upon the MoSCoW method, which is a technique used in business analysis as well as in software development. Its purpose is to set the importance level of various criteria needed for the project. \ref{moscow}. However, in this MoSCoW analysis the ``Would like'' part is left out, as it has no relevance. \\

The features of our lanaguage can be clasified using MoSCoW analysis in three basic classes:
\begin{itemize}
\item \textbf{Must have}: This category shows the most necessary features in the source language. These features are the ones that can not be overlooked in the source language, as it can have a great impact on whether or not the source language will
work correctly. 

\item \textbf{Should have}: This category shows the features that should be implemented in the source language, however this category can be left out, without having a huge impact on the functionality of source language. 

\item \textbf{Could have}: This category is features that could be nice to have in the source language, but is only there to increase luxury in programming in the source language, so leaving these out will have minor impact on how it works. 
\end{itemize}


For this project the following have been chosen:\\

\textbf{Must have:}
\begin{itemize}
\item If-statement \\
The feature of choosing one path over another is so common to programming that it would make no sense to leave it out. One must be able to choose what to do based on something important to the program. \\

\item \textbf{Expressions} \\
In relation to if-statements it is impossible to leave out the ability to evaluate expressions to true or false. \\

\item \textbf{Functions, parameters and return values} \\
The ability to reuse code can reduce the cluttering of code significantly, and help minimizing errors, by allowing the use to reuse the same code, in stead of typing it
multiple times. While it may be possible to leave out functions, it will not make sense from a user-friendly perspective. Making use of parameters and return values removes the complexity of passing by reference. \\

\item \textbf{While loop} \\
Another very useful feature is to be able to repeat the same task multiple times. Instead of having to write the code multiple times it is more efficient to simply run the same piece of code. \\

\item \textbf{Blocks (do/end signs)} \\
Making it clear for the user where a block begins and ends helps reduce confusion. For example, it might be hard to figure out where a while loop ends without an ``end'' keyword or some other marker. \\

\item \textbf{Print} \\
The ability to print out the results of a program is an easy way for the programmer to see the outcome. \\

\item \textbf{Logical operators} \\
Being able to select and group operators is important to perform decision making in programs. This is a necessary feature of any general purpose language.\\

\item \textbf{Simple mathematical operators} \\
Leaving out mathematics removes a lot of functionality and heavily reduces the options of programming. \\

\item \textbf{Type definitions} \\
The user has to be capable of identifying what is being used. Without clearly defined types to use, it can be very difficult for a beginner to understand how to program. \\

\item \textbf{Imperative language} \\
Choosing Imperative language means that the user will have to write code, that describes in exact details what steps has to be done and what is the next step. The Imperative language are supported by most of the mainstream object-oriented programming languages such as Java and C\#. \ref{Impr}

\item \textbf{UTF-8} \\
UTF-8 is short for \textbf{U}CS(Universal Character Set) \textbf{T}ransformation \textbf{F}ormat-8bit. It can represent every character in the Unicode set and provides compatibility with ASCII. 
The reason UTF-8 is a must have is because the typeset is very broad. This means that the user does not have to worry about whether or not his special characters are acceptable, which makes it easier for the programmer to write code.\\
\end{itemize}

\textbf{Should have:}
\begin{itemize}
\item Initialization \\
There are different ways that initialization can work depending on the language as well as the type of the object. \\
\end{itemize}

\textbf{Could have:}
\begin{itemize}
\item User input \\
While definitely a feature which is relevant for some purposes, the feature is not required in order to have a programming language able to use the features of Arduino. \\

\item \textbf{Advanced mathematical operators} \\
Mathematics can be solved without the use of advanced mathematic operators such as square root and modulus, but having the features is definitely convenient. \\

\item \textbf{Loops (Other than while)} \\
The ``while-loop'' has all the functionality needed, but in some cases there are more intuitive ways to go around a problem. ``Foreach'' is a good example of this problem.\\
\end{itemize}