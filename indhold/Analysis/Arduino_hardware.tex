
\section{The physical Arduino}
An Arduino is a small board, a single-board micro-controller.

The board were created to make a cheap way to control student-built interaction design projects. The board consist of open source hardware, which is designed around an 8-bit Atmel AVR micro-controller. The Arduino boards varies in sizes, but taking the Arduino Uno board as an example, then the board has a max width of 2.1'' and a length of 2.7''. 2.1'' is about 5,33cm and 2.7'' is about 6,86cm on the metric scale. 

The board provides some input and output possibilities, however this varies depending on the board, though most have 14 digital I/O and 6 analog inputs. The I/O functions is placed on top of the board, free accessible, and consists of 0.1'' female headers. Besides the I/O there is also Power connector, which is almost all cases will require 5 volt DC. There is an USB connection on the board, so that processing data to the micro-controller is possible, though it is shown as a virtual com-port on the connected computer. However on older boards, instead of the USB connection, a RS232 were used for serial communication. 

On the board there is an LED diode which is connected to the digital pin 13. When this diode is set to ``HIGH'' it will be turned on, and if its value is ``LOW'' it is off. Besides the LED diode there is also a reset button. If this button is pressed the micro-controller is reset. 

The Arduino board gain extra features through shields. Shields is a ``board'', or rather an expansion board, which is plugged onto the Arduino board. These can provide the Arduino board with the possibilities to control for example motors, sensors and LCD displays. This increases the many possibilities there is with an Arduino board. 

kilder:\todo{Jonas: ZAT!! DET ER FOR DOVENT AT SMIDE KILDER HER!}
\todo{Jonas: Sectionen bør nok hedde ``The Arduino board''. Og et billede ville ikke gøre skade.}

http://en.wikipedia.org/wiki/Arduino
http://en.wikipedia.org/wiki/File:UnoConnections.jpg
http://arduino.cc/en/Main/ArduinoBoardUno
