\section{The hardware components of Arduino}
Arduino is a single-board micro-controller.
The role of the board is to make a cheap way to control student-built interaction design projects. A board consists of open source hardware, which is designed around an 8-bit Atmel AVR micro-controller. Arduino boards varies in sizes. Arduino Uno board for example, has a max width of 2.1'' (5,33cm) and a length of 2.7'' (6,86cm). 

The board provides some input and output possibilities. However these varies depending on the board, though most have 14 digital I/O and 6 analog inputs. The I/O functions are placed on top of the board, are freely accessible, and consist of 0.1'' female headers. Besides the I/O there is also a Power connector, which almost in all cases require 5 volt DC. There is an USB connection on the board, so that processing data to the micro-controller is possible, though it is shown as a virtual com-port on the connected computer. However, on older boards, instead of the USB connection, a RS232 were used for serial communication. 

On the board there is an LED diode which is connected to the digital pin 13. When this diode is set to ``HIGH'' it will be turned on, and if its value is ``LOW'' it turns off. Besides the LED diode, there is also a reset button. If the button is pressed the micro-controller is reset. 

Arduino board gain extra features through shields. Shields is a ``board'', or rather an expansion board, which is plugged onto the Arduino board. This provides Arduino boards with the possibilities to control extra components for example motors, sensors and LCD displays.

kilder:\todo{Jonas: Cites please}
%
%http://en.wikipedia.org/wiki/Arduino
%http://en.wikipedia.org/wiki/File:UnoConnections.jpg
%http://arduino.cc/en/Main/ArduinoBoardUno
