This section will look at the testing of the different aspects of the process, and describe these how tests were done. The focus of the tests will be on lexical analysis and the compiler.

\subsection*{Testing method}
There are a wide variety of methods for testing computer software, these can be broken into two main groups, white box and black box testing. The difference between the two are, that with black box test the focus is not on what happens inside the program, but rather look at the program as a box that receives an input and returns an output based on the input. whereas with white box testing, the focus is as much on how the code is running as on the output.\\

The Advantages of the two different methods are:
\begin{itemize}
\item[] \textbf{Black box} It is not necessary to have any knowledge of the code, the tester only needs to know how to pass the input parameter. This also makes it faster to run and evaluate the tests, since the 
\item[] \textbf{White box} Since it tests the code, it is possible to use these tests to optimize the code, although this kind of testing requires a expert knowledge of the code to design the tests. 
\end{itemize}   

\subsection*{Syntax}
\begin{table}[thp]\scriptsize
\centering
\begin{tabular}{|l|l|c|}
\multicolumn{1}{c}{Test case} &
\multicolumn{1}{c}{Result} &
\multicolumn{1}{c}{} \\
\hline
{\begin{lstlisting}[numbers=none,frame=none,resetmargins=true]
void setup() do
end
\end{lstlisting}} & 
{\begin{lstlisting}[numbers=none,frame=none,resetmargins=true]
void setup(){
}
\end{lstlisting}} &
\checkmark\\
  \hline
{\begin{lstlisting}[numbers=none,frame=none,resetmargins=true]
instantiate LiquidCrystal lcd(12,11,5);
CALL lcd.print("Hello World"); 
\end{lstlisting}} & 
{\begin{lstlisting}[numbers=none,frame=none,resetmargins=true]
LiquidCrystal lcd(12,11,5);
lcd.print("Hello World");
\end{lstlisting}} &
\checkmark\\
\hline
{\begin{lstlisting}[numbers=none,frame=none,resetmargins=true]
if(true)do
	x = 1;
end
else do
	x = 2;
end 
\end{lstlisting}} & 
{\begin{lstlisting}[numbers=none,frame=none,resetmargins=true]
if(true){
	x = 1;
}
else {
	x = 2;
} 
\end{lstlisting}} &
\checkmark\\
\hline
\end{tabular}
\caption{Examples of test cases}
\label{tab:test}
\end{table}

The test case shown in table \ref{tab:test} are just some examples of the case the were run during the development of the compiler, also there were more complex cases, which have been omitted due to readability. Examples of these more complex test cases are the complete code examples found in chapter \ref{sec:code_examples}, code example \ref{lst:syntax1} and \ref{lst:syntax2}\\

The tests have been conducted repeatedly through the process every time a new feature or some new functionality have been completed to help make sure that not only the does the new implementations work as intended, but also that it does not result in some other function not working properly.

\subsection*{Type checking}

\subsection*{Functionality}
