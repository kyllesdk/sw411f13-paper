\chapter{JavaCC - Java Compiler Compiler}
JavaCC is a parser generator that generates a fully functioning parser. JavaCC takes a file written using EBNF grammar and makes a program that can recognize matches between the input code and the grammar, this part of the parser is the scanner. JavaCC also creates an AST (Abstract Syntax Tree, see section \ref{sec:AST}) that describes the structure of the derivation of the code. \cite{JavaCC}

JavaCC generates top-down parsers (LL(k) parsers), which means that it replaces every nonterminal symbol until the string is created. LL(k) grammars produce a left-to-right symbol scan which means that it produces a leftmost derivation. The LL(k) parsers use a maximum of \textit{k} symbols of lookahead.

The process of a parser program:
\begin{itemize}
\item Inputs a set of token definitions, grammar and actions
\item Outputs a Java program which performs lexical analysis, meaning that:
	\begin{itemize}
	\item Finding tokens.
	\item Parses the tokens according to the grammar.
	\item Executes actions.
	\end{itemize}
\end{itemize}


\section{Grammar example}
Code table \ref{lst:javacc-grammar-example} contains an example of how a grammar for an ``if''-statement can be constructed. The grammar in the example states the following:

\begin{itemize}
	\item Skip all whitespace and newlines.
	\item An ``if statement'' consists of a condition (C), a statement (S)  and optionally an ``else statement'' which contains a Statement (S)·
	\item A condition (C) consists of the string ``TBD'' - not anything else.
	\item A statement (S) also consists of the string ``TBD'' or a new ``if''-statement
\end{itemize}

\begin{lstlisting}[caption={One of JavaCC's standard examples on how to make a grammar that accepts ``if''-statements.}, label=lst:javacc-grammar-example]

PARSER_BEGIN(Example)

public class Example {

  public static void main(String args[]) throws ParseException {
    Example parser = new Example(System.in);
    parser.IfStm();
  }

}

PARSER_END(Example)

SKIP :
{
  " "
| "\t"
| "\n"
| "\r"
}

void IfStm() :
{}
{
 "if" C() S() [ "else" S() ]
}

void C() :
{}
{
  "TBD"
}

void S() :
{}
{
  "TBD"
|
  IfStm()
}
\end{lstlisting}

Using JavaCC a parser is generated that is used in the final compiler to determine wether the source language is written correctly or not. The code generation part of the compiler (more info in section \todo{ref here}) uses the stream of tokens sent from the parser to specify how to translate the source language into the target language. 