\section{Lookahead} \label{sec:lookahead}
A parser can in some cases require a feature called lookahead. This feature allows for choosing the right path in a grammar, where it is not possible to determine the correct outcome immediately. Such points leading to different paths can for example be represented by brackets in the grammar which are optional parts. 
A part of a grammar could be $``a'' ``b'' [``c'']$, an acceptable string can be a $``ab''$ or $``abc''$. In this case the parser will first look for $``a''$, then for $``b''$. Then it has to decide whether it is done reading (The string is $ab$) or if it should continue, making the string $abc$. Assuming that the string $abc$ is matched, if this grammar function now requires a $``c''$, the parser will make a mistake in matching $abc$ too early, and must back-trace all the way to the chosen point to take the other option. 
Back-tracing is however inefficient and undesirable which is why lookahead was chosen. Instead of blindly choosing one path over another lookahead allows to look ahead in the grammar and make qualified decisions. JavaCC uses this feature and does not back-trace \todo{Morten: Source needed}. By default javaCC uses lookahead(1) which is what is used in BAL.