\section{Lookahead} \label{sec:lookahead}
A parser can in some cases require a feature called lookahead. This feature allows for choosing the right path in a grammar, where it is not possible to determine the correct outcome immediately. Such points leading to different paths can, for example, be represented by brackets in the grammar which are optional parts. 

A part of a grammar could be $``a'' ``b'' [``c'']$, an acceptable string can be a $``ab''$ or $``abc''$. In this case the parser will look for $``a''$, then for $``b''$. Then it has to decide whether it is done reading (the string is $ab$), or if it should continue making the string $abc$. Assuming that the string $abc$ is matched. The parser will then have made a mistake in matching $abc$ too early and must back-trace all the way to the chosen point to take the other option. 

Back-tracing is however inefficient and undesirable, which is why lookahead was chosen. JavaCC uses this feature and does not use back-traces. JavaCC uses lookahead(1) by default, which is also what is used in BAL. The most important reason to use lookahead(1) in BAL, is because the parsers need to check if an if-statement contains an elseif-statement. 