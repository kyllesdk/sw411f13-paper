\subsection{Typechecking}
To make sure the programmer does not mix the different data type variables when writing a program in PH typechecking used.

In listing \ref{list:type_bool} a typechecker visitor is used to check that a boolean variable is set to either ''1`` (true) or ''0`` (false). This is done with typechecking rather than the lexical analysis to make sure that any positive number does not evaluate to true, when using a boolean data type.
\begin{lstlisting}[Caption=Verifies/falsifies whether a boolean variable is 0 or 1, label=list:type_bool]

    public Object visit(ASTBooleanNumber node, Object data) {
    	String number = node.value.toString();

		if("(0)".equals(number) || "(1)".equals(number)) {
			
			return DataType.TypeBoolean;
		} else {
			System.out.println("Type error: A boolean number has to be 1 or 0");
			
			return DataType.TypeNotImportant;
		}	 
    }
\end{lstlisting}

When using arithmetic expressions on floating point numbers and integers together it might result in issues. For instance when a floating point number is added to an integer it is likely to lose precision. On the other hand if an integer is added to a floating point number the precision will not change as they already have a higher precision. As seen in listing \ref{list:type_float} an error message is shown if the first child is of the integer type and the second child is of the float type. Additionally for if any variable is declared to a boolean data type an error message will appear, telling the programmer that a boolean variable is expected.

\begin{lstlisting}[Caption=Checks if float is being added to an integer\, and if a non-boolean type is set to a boolean value, label=list:type_float]
	public Object visit(ASTDeclaration node, Object data) {
		PrintVisitor pv = new PrintVisitor();

		if(((DataType)node.jjtGetChild(0).jjtAccept(this, data) == DataType.TypeInteger) && ((DataType)node.jjtGetChild(1).jjtAccept(this, data) == DataType.TypeFloat)) {

			_errorMessage = "Trying to add Float to Integer, this result in loss of precision.";

			return DataType.TypeUnknown;

		} else if(((DataType)node.jjtGetChild(0).jjtAccept(this, data) != DataType.TypeBoolean) && ((DataType)node.jjtGetChild(1).jjtAccept(this, data) == DataType.TypeBoolean)) {

			_errorMessage = "Expected the variable to be of type boolean.";

			return DataType.TypeUnknown;
		}

		return DataType.Declaration;
	}
\end{lstlisting}

