\section{Call-by methods}
Three different methods are used in our language to pass arguments, named call-by-value, call-by-reference and call-by-name. When a function is called there can optionally be parameters involved which could be passed with any of these three methods.\\
Call-by-value is used by copying the value of some argument. Let for example a function $f$ call the integer $i$ as a parameter. The function does however not alter the variable $a$, instead it creates a copy local to the function holding the value of $i$, hence it is called by its value. Furthermore, any operations will be performed before the call-by-value meaning that the pass holds the evaluated result of an argument rather than the argument. As an example, if $x$ was an integer with the value $41$ passing $x+1$ to a function will hold the evaluated result of the calculation - $42$. \\
Call-by-reference is used to always modify the same variable rather than copying values. With call-by-value a function could use a copy of some variable as input and modify this copy in any way without ever affecting the original variable. Call-by-reference differs in that it uses the original variable, meaning no copy is made and any changes a function makes to this variable are also changed outside that context. By example, if the integer $i = 41$ is called-by-reference by a function $f$, this function will be able to reference $i$ in any operations. As such the simple operation $i = 0$ will reset $i$ to the value $0$ even though it is not a variable of the function. \\
Call-by-name is less obvious than call-by-value and call-by-reference. Basically this method allows the passing of expressions without evaluating them. Using the previous example with call-by-value, passing $x+1$ would be evaluated first and then passed as $42$. Call-by-name is different in that it passes the expression itself. The expression will not be evaluated until it is accessed in the function. Furthermore it will be evaluated on every call. As such the pass to the function holds $x+1$, and only when accessed will the expression be evaluated to $42$. Unique to call-by-name, this evaluation happens every time it is accessed. This allows $x$ to be evaluated correctly every time, even if it should somehow be assigned to a new value.\\
\todo{Morten: Do we really pass by name?}