\section{Implementation of the Scanner}
When creating the lexical analyzer part of the parser generation, context free grammar is used in the form of EBNF (see Section ??). In this section a few of the EBNF implementations will be described.
\begin{lstlisting}[label=arit_listing]
void ArithmeticExpression() :
{}
{
	(<PLUS> | <MINUS> | <MULTIPLY> | <DIVIDE> | <MOD> | <SQRT> | <POW>)

}

\end{lstlisting}
One of the features for the source language is basic arithmetic expressions. These different expressions have been implemented as seen in Listing \ref{arit_listing}. Since the lexical analyzer reports error whenever there is a wrong use of syntax, \todo{tror der er fejl i det her kode}.


\begin{lstlisting}[label=if_listing]
void IfStatement() :
{}
{
    <IF> <LPAREN> BooleanExpression() <RPAREN> <DO> Statement() <END> [<ELSEIF> <LPAREN> Expression() <RPAREN> <DO> Statement() <END>] [<ELSE> <DO> Statement() <END>]
}
\end{lstlisting}
Another feature is the use of if statements, but with the opportunity to use only the if statement, follow up with the amount of elseif statements the programmer desires, or end with an else statement. This has been implemented as seen in listing \ref{if_listing}, most importantly to note is the ability to use elseif zero or many times with the use of a regular expression. To make sure the order of the if statements are created in the right order, having an if statement before an elseif or else is required \todo{Gør vi det? Og hvordan}. To make sure there is no dangling else problems, every if, elseif, or else statement requires a "do" "end" block to avoid ambiguity.