\subsection{Translation}
In this subsection the implementation of the translator is described. This will include examples of how translation from PH to the Arduino language is done through visitors.
Visitors will scan through the code till it finds the appropriate statement or expression to translate.

\begin{lstlisting}[caption=Visitor for translation of an if-statement, label=list:if]

	public Object visit(ASTIf_stm node, Object data) {
		System.out.print("if (");
		node.jjtGetChild(0).jjtAccept(this, data);
		System.out.println(") {");
		node.jjtGetChild(1).jjtAccept(this, data);
		System.out.println("");
		System.out.println("} ");
		if(node.jjtGetNumChildren() > 2) {
			for(int i = 2; i <= node.jjtGetNumChildren(); i++) {
				if(i < (node.jjtGetNumChildren() -1)) {
					node.jjtGetChild(i).jjtAccept(this, data);
				} else {
					node.jjtGetChild(i).jjtAccept(this, data);
					break;
				}
			}
		}

\end{lstlisting}

As seen in listing \ref{list:if}, a CodeGen visitor is taken in use to translate an if-statement in PH to the equivalent in Arduino. The changes depending on the code consist of translating, for example:\\
\begin{itemize}
\item ''do`` and ''end`` blocks are translated to ''{`` and ''}`` braces.
\item operators, such as "EQUALS" are translated to "==". 
\end{itemize}
If there are more than two children, the translator iterates through them until they are all checked. This ensures that in the event of an if statement containing more than one expression, all expressions are included in the translation process.\\
\begin{lstlisting}[caption=Example of an if-statement written in PH, label=list:phstmt]

#include <LiquidCrystal.h>

void setup()do
 int h = 14;
end

void loop()do
 if(b EQUALS z)do
  int x = 3;
  string hello = "world";
 end
end
\end{lstlisting}

\begin{lstlisting}[caption=Example of an if-statement in the Arduino language translated by the compiler from PH, label=list:stmtardu]

#include <LiquidCrystal.h>

void setup(){
 int h = 14;
}

void loop(){
 if(b == z){
  int x = 3;
  string hello = "world";
 }
}
\end{lstlisting}

As seen in \ref{list:phstmt}, an example of an if-statement containing more than one expression is written. When translated by the compiler the output is as seen in \ref{list:stmtardu}.




