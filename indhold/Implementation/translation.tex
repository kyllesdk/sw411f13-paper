\section{Translation}
In this subsection the implementation of the translator is described. This will include examples of how translation from BAL to the Arduino language is done through visitors.
Visitors will scan through the code till it finds the appropriate statement or expression to translate.

\begin{lstlisting}[caption=Visitor for translation of an if-statement, label=list:if]

	public Object visit(ASTIf_stm node, Object data) {
		System.out.print("if (");
		node.jjtGetChild(0).jjtAccept(this, data);
		System.out.println(") {");
		node.jjtGetChild(1).jjtAccept(this, data);
		System.out.println("");
		System.out.println("} ");
		if(node.jjtGetNumChildren() > 2) {
			for(int i = 2; i <= node.jjtGetNumChildren(); i++) {
				if(i < (node.jjtGetNumChildren() -1)) {
					node.jjtGetChild(i).jjtAccept(this, data);
				} else {
					node.jjtGetChild(i).jjtAccept(this, data);
					break;
				}
			}
		}

\end{lstlisting}

\begin{lstlisting}[caption=Visitor for translation of an boolean operators, label=list:bool]
	public Object visit(ASTBool_op node, Object data) {
		node.jjtGetChild(0).jjtAccept(this, data);
		System.out.print(" ");

		String nodeValueString = node.value.toString();
		if("AND".equals(nodeValueString)) { System.out.print("&& "); }
		else if("EQUALS".equals(nodeValueString)) {System.out.print("== ");}
		else if("NOT".equals(nodeValueString)) {System.out.print("!");}
		else if("NOTEQUALS".equals(nodeValueString)) {System.out.print("!= ");}
		else if("OR".equals(nodeValueString)) {System.out.print("|| ");}

		node.jjtGetChild(1).jjtAccept(this, data);

		return data;
	}
\end{lstlisting}

As seen in listing \ref{list:if} and  \ref{list:bool}, CodeGen visitors are taken in use to translate an if-statement in BAL to the equivalent in Arduino. For an if-statement, if there are more than two children, the translator iterates through them until they are all checked. This ensures that in the event of an if statement containing more than one expression, all expressions are included in the translation process.\\
\begin{lstlisting}[caption=Example of an if-statement written in BAL, label=list:phstmt]

#include <LiquidCrystal.h>

void setup()do
 int h = 14;
end

void loop()do
 if(b EQUALS z)do
  int x = 3;
  string hello = "world";
 end
end
\end{lstlisting}

\begin{lstlisting}[caption=Example of an if-statement in the Arduino language translated by the compiler from BAL, label=list:stmtardu]

#include <LiquidCrystal.h>

void setup(){
 int h = 14;
}

void loop(){
 if(b == z){
  int x = 3;
  string hello = "world";
 }
}
\end{lstlisting}

As seen in listing \ref{list:phstmt}, an example of an if-statement containing more than one expression is written. When translated by the compiler the output is as seen in listing \ref{list:stmtardu}. The visitor for the if-statement translates the ''do`` and ''end`` blocks into ''{`` and ''}`` braces. And the boolean operator visitor will in this case translate the operator ''EQUALS`` into ''==``.