\subsection{Translation}
In this subsection the implementation of the translator is described. This will include examples of how translation from PH to the Arduino language is done through visitors.
Visitors will scan through the code till it finds the appropriate statement or expression to translate.

\begin{lstlisting}[Caption=Visitor for translation of an if-statement, label=list:if]

	public Object visit(ASTIf_stm node, Object data) {
		System.out.print("if (");
		node.jjtGetChild(0).jjtAccept(this, data);
		System.out.println(") {");
		node.jjtGetChild(1).jjtAccept(this, data);
		System.out.println("");
		System.out.println("} ");
		if(node.jjtGetNumChildren() > 2) {
			for(int i = 2; i <= node.jjtGetNumChildren(); i++) {
				if(i < (node.jjtGetNumChildren() -1)) {
					node.jjtGetChild(i).jjtAccept(this, data);
				} else {
					node.jjtGetChild(i).jjtAccept(this, data);
					break;
				}
			}
		}

\end{lstlisting}

As seen in listing \ref{list:if} a codegen visitor is taken in use to translate an if-statement in PH to the equivalent in Arduino. The changes consist of translating braces ''{``, ''}`` to ''do`` and ''end`` blocks. The children 