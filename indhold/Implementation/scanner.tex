\section{Lexical Analysis}
\subsection{Scanner}
This section contains a description of the implementation of the scanner in the compiler. The scanner generates a token stream by reading the input one character at a time.\\
\\The implementation of the scanner contains a list of token identifiers. Token identifiers recognize when a token is found in the process of scanning the input. Token identifiers for the reserved words and literals in the source language can be seen in listing \ref{list:token}. For example, when "setup()" is scanned, it is saved as a SETUP token. The token will then be used in the parser task of the lexical analysis. 

\begin{lstlisting}[caption=Token identifiers for reserved words and literals, label=list:token]
TOKEN : /* Reserved words and literals */
{
	< SETUP : "setup()" >
|	< LOOP: "loop()" >	
|	< CALL: "call" >
|	< INSTANTIATE: "instantiate" >
|	< VAR: "var" >
|	< BOOLEAN: "boolean" >
|	< DOUBLE: "double" >
|	< ELSE: "else" >
|	< ELSEIF: "elseif" >
|	< FLOAT: "float" >
|	< IF: "if" >
|	< INT: "int" >
|	< NULL: "null" >
|	< STRING: "string" >
|	< WHILE: "while" >
| 	< DO: "do" >
| 	< END: "end" >
|   < VOID: "void" >
|   < FUNCTION: "function" >
|   < RETURN: "return" >
}

\end{lstlisting}
Since JavaCC uses EBNF, BAL supports token identifiers defined with regular expressions. As seen in listing \ref{list:spectoken}, regular expression is used to define single line comment tokens. It is instructed to ignore carriage returns (\textbackslash{r}) and line feeds (\textbackslash{n}), zero or more times. 

\begin{lstlisting}[caption=Special token for single line comment, label=list:spectoken]
SPECIAL_TOKEN :
{
  <SINGLE_LINE_COMMENT: "//" (~["\n", "\r"])* ("\n" | "\r" | "\r\n")?>
}
\end{lstlisting}


 
