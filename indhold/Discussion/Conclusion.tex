\section{Conclusion}

In this project we have proposed a language for Arduino. The main features of PH are increased writeability, readability, and intuitivity. The language is suppose to be an alternative to the Arduino language, and not a substitute.\\

PH is made simpler by removing the feature of being able to produce an outcome in several different ways. For example, loops have been limited to while, and numerals have been limited to integers and floats. Removing parts of the Arduino language has resulted in a simple language, consequently giving the programmer a better overview of the syntax, and making it easier to learn. Different from the Arduino language, PH does not have the need to encapsulate statements inside if-statements, functions, and loops with braces. Instead PH achieves more readability and intuitivity by using the keywords ''do`` and ''end``. This approach is also used in boolean expressions where keywords such as ''and`` and ''or`` are used rather than ''\&\& `` and ''||``. The similarity to English makes PH more intuitive to learn and use because it is similar to an already known syntax. As such the overall goals have been achieved. \\
The task of writing a translator which converts a program from PH to the Arduino language has been achieved, and the process is documented in the report. The validity of the translation is verified by the tests performed in
chapter \ref{chap:test}. Having performed a lot of white-box examples using various code examples, it is fairly reasonable to conclude that the compiler works as intended. The one error that did persist was proven not to be related to the compiler and is as such irrelevant for the purpose of this project. Although we did not encounter any errors in the compiler it is not a guarantee that there is no errors at all. While the white-box testing went well it would require further tests to verify that the compiler is completely bugfree. Using unit-testing to break the compiler into small parts and verifying that each part works as intended would be a good way to do this, but also a time-consuming method. \\
As mentioned in section \ref{discussion:improvements} functions were not included in the solution. As described in the MoSCoW analysis, this is considered to have great a impact on the language. However, PH can make function calls to execute statements, if they are included in the Arduino language. Furthermore, only while loop was implemented, because it has the same possibilities as a ''for`` loop. \\

PH has achieved the preliminary goals, however, there are improvements that can be implemented in a further development.\\
\todo{Morten: Mangler muligvis noget om Vikings hardware error}