\section{Conclusion}

In this project we have attempted to create a language for Arduino as described in \ref{analysis:source-language}. The main target features are that it should be simpler and easier to use than the official language along with being intuitive, even for beginners. \\
The resulting language is stripped from duplicate methods of producing similar outcomes. For example, loops has been limited to while, and variables containing numbers has been limited to only include integers and floats. This is seen in \ref{analysis:informal-specification}. Removing redundant parts of the language has resulted in a smaller language, consequently making the new language easier to overview. Different to the official Arduino language, our language does not have the need to parenthesize elements such as if-statements or loops. Instead the language achieves more readability and intuitivity by using the keywords do and end. This approach is also used in boolean expressions where keywords such as and and or are used rather than \& \& and  ||. The similarity to the English language makes the language more intuitive to learn and use because it is more similar to a language that any English-speaking people are familiar with. As such the overall goals has been achieved to some degree. \\
The task of writing a translator which can convert a program from our language to Arduino has been achieved and the progress documented by the report. The validity of the translation is verified by the tests performed in \todo{Morten: Ref til test}. The tests show that blablabla. \\
Comparing the finished project with the MoSCoW model described in \ref{analysis:moscow}, every feature within the category of must have has been achieved with the exception of functions (Including parameters and return values). As described in the MoSCoW definition of must have, this has great impact on the language. The resulting language is impaired in that it is unable to make calls to execute statements, meaning that code can not be reused. This makes the language unsuited for any programming where this functionality is desired. \\
Of the less important features, the ability to print along with advanced operators has been implemented. We did however not get around to implement other loops than while, even though a foreach loop was desired. This does however not impair the functionality of the language and is a matter of convenience. \\
The language does not currently respect reserved words in Arduino language which is something that should be solved as using these words will result in an error when compiling from Arduino to machine code. The language does not deal with the need for parenthesization in arithmetic expressions and leaves this job to the Arduino compiler.

In conclusion, the language has achieved the base goals set for it. It does however feel unfinished, and could be developed further for considerable improvements. While the product demonstrates the philosophy of the new language well, the solution is not optimal as a programming language without improvements.\\

WORK IN PROGRESS

 - Arduino hardware error \\
 - Results from testing \\